The following is a summary of some of the work that has been done in
the field of automatic whisker tracking, as well as a Ph.D thesis on
probabilistic tracking of human motion. The work in this thesis will
be based on the latter since the whisker tracking problem is similiar,
though in many regards simpler.


In 2011, Roy et al. \cite{BadExample1} describe a whisker tracking
system that uses motion capture markers and two tracking cameras to
track whisker movements in 3D. A spatial resolution of $<0.5$ mm in
all dimensions and a temporal resolution of $5$ ms is reported. The
impact of the markers on whisker movements were investigated by
comparison with a light beam detection system, and no significant
difference was detected. The system requires head fixation since the
markers need to be visible to both cameras at all times.
        

In 2008, Voigts et al. \cite{UnsupervisedTracking} developed a system
that uses frame-by-frame image analysis on off-line monocular images,
and does not impose other restrictions on the setup. The solution is
based on creating vector fields using anisotropy in the image and
tracks movements along the full length of the whiskers. It is fully
automatic, and successfully tracks up to 8 whiskers on each side of
the snout simultaneously, though it suffers some difficulties when
applied on full whisker arrays.


In 2001, Hedvig Sidenbladh \cite{Hedvig} investigated the general
problem of tracking 3D human motion in monocular video without making
assumptions about the appearance of the human or environment. The
thesis sets up a probabilistic framework and combines multiple visual
cues, and achieves good accuracy in tests. It has been used throughout
the work on this thesis as inspiration and a reference on
probabilistic tracking.
