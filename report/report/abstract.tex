The interest in studying rodent whiskers has recently seen a significant increase, 
particularly in the field of neurophysiology. As a result, there is a need for automatic 
tracking of whisker movements. Currently available commercial solutions either are 
extremely expensive, restrict the experiment setup, or fail in the presence of 
\emph{clutter} or occlusion. 

This thesis proposes a proof-of-concept implementation of a probabilistic tracking system. 
This solution uses a technique known as the \emph{Particle Filter} to propagate a whisker model between frames of high speed video. 
In each frame, the next state of the model is predicted by querying a pre-trained database and filtering the results 
through the Particle Filter. The implementation is written in Python using NumPy and an SQLite3 database.

Testing results indicate that the approach is feasible. Tests on real whisker
videos 
First, it successfully tracks multiple whiskers at once, even under clutter. 
Second, being a standalone program operating on pre-recorded video, it does not notably restrict the experiment.\\

\textbf{Keywords:} Tracking, Multiple, Whisker, Particle Filter, Transition Database, Model Evaluation, Proof-of-Concept
