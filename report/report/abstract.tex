The interest in studying rodent whiskers has recently seen a
significant increase, particularly in the field of neurophysiology. As
a result, there is a need for automatic tracking of whisker
movements. Currently available commercial solutions either are
extremely expensive, restrict the experiment setup, or fail in the
presence of clutter or occlusion.

This thesis proposes a proof-of-concept implementation of a
\emph{probabilistic} tracking system.  This solution uses a technique
known as the \emph{particle filter} to propagate a whisker model
between frames of high speed video.  In each frame, the next state of
the model is predicted by querying a pre-trained \emph{database} and
filtering the results through the particle filter. The implementation
is written in Python 2.6 using NumPy and SQLite3.

Testing results indicate that the approach is feasible. Even using a
rather crude database, the tracker manages to track multiple real
whiskers at once, though only for short sequences at a time. Better
training data, such as hand-labeled real data, might vastly improve
the result.

\textbf{Keywords:} Tracking, Multiple, Whisker, Particle Filter,
Transition Database, Model Evaluation, Proof-of-Concept
