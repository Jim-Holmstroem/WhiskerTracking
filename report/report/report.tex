\documentclass[a4paper,11pt]{kth-mag}
\usepackage[pdftex]{graphicx}
\usepackage[T1]{fontenc}
\usepackage{textcomp}
\usepackage{lmodern}
\usepackage[utf8]{inputenc}
\usepackage[swedish,english]{babel}
\usepackage{modifications}
\usepackage[retainorgcmds]{IEEEtrantools}
\usepackage{amsmath}
\usepackage{amsthm}
\usepackage{amssymb}
\usepackage{amsfonts}

\usepackage{clrscode}
\usepackage{hyperref}
\usepackage{graphicx}
\usepackage{tikz}
\usepackage{pgfplots}

\graphicspath{
        {figures/}
        {figures/smiley/}
        {figures/mp-graph/}
        {figures/hmm-graph/}
        {figures/hmm-pf-graph/}
        {figures/gwhisker_database/}
        {figures/gwhiskers/}
        {figures/COD/}
        {figures/response/}
        {figures/preprocessing/}
        {figures/snout_tracking/}
        }

\theoremstyle{definition}
\newtheorem{definition}{Definition}
\newtheorem{example}{Example}
\newtheorem{theorem}{Theorem}
\newtheorem{note}{Note}

\newcommand{\ForEach}{\textbf{for each} }
\newcommand{\In}{\textbf{in} }
\newcommand{\Spline}[2][a]{#1_3#2^3 + #1_2#2^2 + #1_1#2}

\newcommand{\NN}{\ensuremath{\mathbb{N}}}
\newcommand{\ZZ}{\ensuremath{\mathbb{Z}}}
\newcommand{\RR}{\ensuremath{\mathbb{R}}}
\newcommand{\CC}{\ensuremath{\mathbb{C}}}
\newcommand{\HS}{\ensuremath{\mathcal{H}}}
\newcommand{\IS}{\ensuremath{\mathcal{I}}}
\newcommand{\XS}{\ensuremath{\mathcal{X}}}
\newcommand{\ZS}{\ensuremath{\mathcal{Z}}}
\newcommand{\TS}{\ensuremath{\mathcal{T}}}

%operators
\newcommand{\abs}[1]{\left\vert #1 \right\vert}
\newcommand{\argmax}[1]{\underset{#1}{argmax}}
\newcommand{\argmin}[1]{\underset{#1}{argmin}}

\newcommand{\prob}[1]{p{(#1)}}
\newcommand{\cprob}[2]{\prob{\left. #1 \middle\vert #2 \right.}}
\newcommand{\cprobnext}[1]{\cprob{#1_t}{#1_{t-1}}}
\newcommand{\ndist}[2]{\mathcal{N}{\left(#1, #2\right)}}
\newcommand{\norm}[1]{\left\vert\left\vert#1\right\vert\right\vert}
\newcommand{\Lp}[1][p]{\mathrm{L}^{#1}}
\newcommand{\Lpnorm}[2][p]{\norm{#2}_{\Lp[#1]}}

\newcommand{\Response}[3]{\langle #1, #2\rangle_{#3}}


\newcommand{\Ordo}[1]{\mathcal{O}{\left(#1\right)}}

\newcommand{\xtmN}[4]{
  \left\{#1_{#2}^{#3}\right\}_{#3=1}^{#4}
}
\newcommand{\interval}[2]{
  \left[#1, #2\right]
}
\newcommand{\bel}[1]{
  bel\left(#1\right)
}

\newcommand{\tf}{x_{\text{from}}}
\renewcommand{\tt}{x_{\text{to}}}
\newcommand{\trans}{\left( \tf, \tt \right)}
\newcommand{\coeffs}{\left( a_3, a_2, a_1 \right)}

\title{
    Probabilistic Tracking of Multiple Rodent Whiskers in Video
    Sequences\\ <SECOND DRAFT>
}

\subtitle{
    A small step towards cheap, reliable, and non-intrusive automatic tracking of whiskers.
}
\foreigntitle{
    Statistisk Följning av Multipla Morrhår hos Gnagare i Video.
}
\author{
    Jim Holmström\\
    \href{mailto:jimho@kth.se}{jimho@kth.se}\\
    Emil Lundberg\\
    \href{mailto:emlun@kth.se}{emlun@kth.se}
}
\date{May 2012}
\blurb{
  SA104X Examensarbete inom teknisk fysik, grundnivå\\
  Bachelor's Thesis at CSC, KTH\\
  Supervisor: Prof. Örjan Ekeberg\\
  Examiner: Prof. Örjan Ekeberg
}
\trita{TRITA xxx yyyy-nn}
\begin{document}
\frontmatter
\pagestyle{empty}
\removepagenumbers
\maketitle
\selectlanguage{english}
\begin{abstract}
    The interest in studying rodent whiskers has recently seen a
significant increase, particularly in the field of neurophysiology. As
a result, there is a need for automatic tracking of whisker
movements. Currently available commercial solutions either are
extremely expensive, restrict the experiment setup, or fail in the
presence of \emph{clutter} or occlusion.

This thesis proposes a proof-of-concept implementation of a
\emph{probabilistic} tracking system.  This solution uses a technique
known as the \emph{particle filter} to propagate a whisker model
between frames of high speed video.  In each frame, the next state of
the model is predicted by querying a pre-trained \emph{database} and
filtering the results through the particle filter. The implementation
is written in Python 2.6 using NumPy and SQLite3.

Testing results indicate that the approach is feasible. Even using a
rather crude database, the tracker manages to track multiple real
whiskers at once, though only for short sequences at a time. Better
training data, such as hand-labeled real data, might vastly improve
the result.

\textbf{Keywords:} Tracking, Multiple, Whisker, Particle Filter,
Transition Database, Model Evaluation, Proof-of-Concept

\end{abstract}
\clearpage
\begin{foreignabstract}{swedish}
    
Intresset för att studera morrhår på gnagare har på senare tid
ökat, speciellt inom neurofysiologin.  Som ett resultat av detta
finns det ett behov av automatisk följning av morrhår.  Nuvarande
kommersiella lösningar är antingen extremt dyra, sätter
begräsningar på experimentuppställningen eller misslyckas i
stökiga miljöer eller när överlappning förekommer.

Denna avhandling bidrar med en enkel implementation av ett
\emph{probabilistiskt} följningssystem.  Lösningen använder sig av
en teknik som kallas \emph{partikelfilter} för att propagera en
morrhårmodell mellan bildrutor i höghastighetsvideo.  För varje
bildruta förutspås nästa tillstånd genom att fråga en förtränad
\emph{databas} och filtera svaret genom partikelfiltret.
Implementationen är skriven i Python 2.6 och använder sig av
programbiblioteken NumpPy och SQLite3.

Testresultaten indikerar att metoden är rimlig.  Med endast en grovt
snarlik databas lyckades algoritmen följa multipla morrhår, dock
endast under korta sekvenser i taget.  Bättre träningsdata, såsom
handmarkerad riktig data, skulle kunna förbättra resultatet
väsentligt.


\textbf{Nyckelord:} Följning, Multipla, Morrhår, Partikelfilter,
Övergångsdatabas, Modelevaluering, Proof-of-Concept

\end{foreignabstract}
\clearpage
\tableofcontents*
\mainmatter
\pagestyle{newchap}
\chapter{Introduction}
    \label{sec:introduction}
    \section{Background}

With the ever increasing power and mobility of computers, computer
vision has recently seen a large increase in interest.  Encompassing
problems such as classification, recognition, perception and tracking,
application of the discipline could make many tasks easier and more
efficient.

In particular, biology and neuroscience researchers are interested in
tracking the movements and posture \cite{WhiskerVideography} of
animals or parts of animals.  One such field aims to study the
movements of rodent whiskers. However, most whisker tracking software
available today suffers a few fundamental flaws.  They are either so
expensive that not even well funded laboratories feel they can afford
them or have problems tracking multiple whiskers at once, often
requiring removal of almost all whiskers.  Some higher precision
systems impose other restrictions on the experiment, such as
restraining the animal or attaching motion capture markers to the
whiskers. \cite{BadExample1} Such restrictions may cause systematic
errors and not give a representative view of how the animal naturally
uses its whiskers.

\section{Difficulties}
In general, the main difficulty in tracking and localization is to
separate the tracked object from clutter and occlusion. In the case of
whisker tracking, the dominant problems are that whiskers often
overlap, and the relatively low spatial and temporal resolution in
highspeed video results in subpixel whiskers and motion
blur. \cite{WhiskerVideography}

In 2001, Hedvig Sidenbladh investigated probabilistic methods for
tracking three-dimensional human motion in monocular
video. \cite{Hedvig} Many of the problems inherent in computer vision
were regarded, and the thesis shows that powerful conclusions can be
drawn by combining multiple visual cues.\footnote{The solution in this
thesis employs only a single visual cue.}

The most apperent problems in whisker tracking is occlusion, 3D to 2D
projection inambiguities and motion blur. Other, more subtle problems
include that the whisker root is constantly occluded by facial
hairs. This gives us the problem of not knowing where the whiskers are
rooted, making the whiskers more difficult to model.

\section{Approache the difficulties}
The problems above will be addressed as they appear in the thesis.

\section{Contributions of this thesis}
The contributions in this thesis can be divided into
\begin{description}
\item[Functional model] Introducing a functional model
  $\text{whisker}$ and using the functional $L_p$-norm, to better
  represent "distance" between two models, for the sampling part of
  the particle filter.
\item[Proof of concept] Providing a proof with examples on that this
  solution is indeed feasible.
\item[Parameter investigation] Parameter analysis on the algorihtm and
  how the parameters effects the result\footnote{Result on the
    benchmark\ref{sec:benchmarks_results}.}.

\end{description}


\chapter{Related Work}
    \label{sec:related_work}
    
\section{Examples of whisker tracking systems}
    \subsection{Unsupervised Whisker Tracking in Unrestrained Behaving Animals\cite{UnsupervisedTracking}}

        Image flow, gabur

        It works well under temporary occlusion but needs clipping of most whiskers and thereby introduce possible effects to the experimental results.



    \subsection{High-Precision, Three-Dimensional Tracking of Mouse Whisker Movements with Optical Motion Capture Technology\cite{BadExample1}}
        


        Uses a 3D tracking system consisting of two high-speed cameras and markers mounted on the whiskers of the head-fixed mouse.
        The head-fixed is for the markers must be visible at all times to make the tracking work.
        

\section{Probabilistic Tracking and Reconstruction of 3D Human Motion in Monocular Video Sequences\cite{Hedvig}}
    




\chapter{Definitions}
    \label{sec:definitions}
    % TODO
% \section{Invariances}
% Section about invariances and why they are needed in localization.

% The "filter" must give the same response invariant of the position
% in the image and its often wanted to have it rotational and scale
% invariant (if your not intending to measure things like angles or
% size that is)

\section{Images and videos}
\subsection{Grayscale image}
\begin{definition}

  A grayscale \emph{image} can be defined as a function
  \begin{equation}
    \begin{array}{ccc}
      I : \NN^2 &\rightarrow& \RR^+\\
      I : \text{position}&\rightarrow&\text{intensity}.
    \end{array}
  \end{equation}
  It can also be identified with $\NN^2\times \RR^+$ as the tuple
  $\langle \text{position}, \text{intensity}\rangle$. The \emph{image
    space} is denoted as $\IS$ in this thesis.

  In a computer an image is represented as a integer matrix, often 8
  bit integers.\footnote{Integers in the range $\interval{0}{255}$.}
\end{definition}

% \subsubsection{Color image}
%A color image, can similary to grayscale images, be defined as
% \begin{equation}
%    \begin{array}{ccc}
%    RGBimage : \NN^2 &\rightarrow& \NN^3\\
%    RGBimage : position&\rightarrow&color
%    \end{array}
%\end{equation}
%Which basically is foreach position in the image we have a response
% color.  An alternitve represenation whould be $\NN^5=\langle
% pos,value\rangle$.
%
%The RGB image space is denoted by
%\begin{equation}
%    RGBimage\in \IS_{RGB}
%\end{equation}

\begin{example}~\\
  \begin{tabular}{lcr}
    $
    \begin{pmatrix} 
      255&  255&  255&  255&  255&  255&  255&  255&  255&  255\\
      255&  255&  255&  255&  255&  255&  255&  255&  255&  255\\
      255&  255&  255&    0&  255&  255&    0&  255&  255&  255\\
      255&  255&  255&    0&  255&  255&    0&  255&  255&  255\\
      255&  255&  255&  255&  255&  255&  255&  255&  255&  255\\
      255&    0&  255&  255&  255&  255&  255&  255&    0&  255\\
      255&  255&    0&  255&  255&  255&  255&    0&  255&  255\\
      255&  255&  255&    0&    0&    0&    0&  255&  255&  255\\
      255&  255&  255&  255&  255&  255&  255&  255&  255&  255\\
      255&  255&  255&  255&  255&  255&  255&  255&  255&  255
    \end{pmatrix}$
    &$\Rightarrow$& \parbox[c]{1em}{\includegraphics[scale=10.0]{smiley.png}}
    \end{tabular}
\end{example}

\subsection{Video}
\begin{definition}
  A \emph{video} is a function mapping an integer to an image:
  \begin{equation}
    \text{video} : \NN \rightarrow \IS.
  \end{equation}
\end{definition}

\section{Image Processing}

\begin{definition}
  The rendering function $R$ takes a hypothesis $x$ and renders an
  image with a resemblance of how a real whisker would have looked
  like having the same underlying model and parameters as $x$.
  \begin{equation}
    \begin{split}
      R : \XS &\rightarrow \IS\\
      x &\mapsto \text{Render}(x)
    \end{split}
  \end{equation}
\end{definition}

\begin{definition}
  The addition operation on images is performed by elementwise
  addition.
  \begin{equation}
    \begin{split}
      + : \IS \times \IS &\rightarrow \IS\\
      (I_a,I_b) &\mapsto I_a+I_b
    \end{split}
  \end{equation}
\end{definition}

\begin{definition}
  The multiplication operation on images is element-wise
  multiplication.
  \begin{equation}
    \begin{split}
      * : \IS \times \IS &\rightarrow \IS\\
      (I_a,I_b) &\mapsto I_a*I_b
    \end{split}
  \end{equation}
\end{definition}

\begin{definition}
  The image transformation $\phi$ takes an image $I$ and returns a
  transformed image of the same size as $I$.
  \begin{equation}
    \begin{split}
      \phi : \IS &\rightarrow \IS\\
      I &\mapsto \text{Transform}(I)
    \end{split}
  \end{equation}
\end{definition}

The structure $\langle \IS,+,*\rangle$ inherites the properties from
$\langle \RR^+,+,*\rangle$ by just being a vectorized version.

\begin{definition}
  We will use subtraction loosely\footnote{Meaning, no analysis on the
    structure is done.} as elementwise subtraction and then subtract
  the smallest element on all elements to make the operation closed.
  \begin{equation}
    \begin{split}
      - : \IS \times \IS &\rightarrow \IS\\
      (I_a,I_b) &\mapsto I_a-I_b
    \end{split}
  \end{equation}
\end{definition}

\begin{definition}
  The image transformation $\phi$ takes an image $I$ and returns a
  transformed image.
  \begin{equation}
    \begin{split}
      \phi : \IS &\rightarrow \IS\\
      I &\mapsto \text{Transform}(I)
    \end{split}
  \end{equation}
\end{definition}

\section{States, hypotheses and estimates}
A system is said to have a \emph{state}. The state is some quantity
that defines the qualities of the system.

\begin{description}
\item[State] The state of a system is denoted $Z$. When time is
  relevant, the state at time $t$ is denoted $Z_t$.
\item[State space] The set $\ZS$ of all possible states, $Z \in \ZS$.
\item[Hypothesis] A guess $x$ at the state $Z$ of a system.
\item[Hypothesis space] The set $\XS$ of all possible hypotheses, $x
  \in \XS$. In general, $\XS \neq \ZS$ since most models are
  simplifications of the system.
\item[Estimate] The hypothesis $x^*$ we believe approximates $Z$ best.
\item[Observation] In general, it is not possible to directly record
  the state $Z$ of a system.\footnote{If it were, there would be no
    need for tracking.} We instead get an \emph{observation} $I$ of
  the state.
\item[Degrees of Freedom] The number of adjustable parameters in a
  model, often abbreviated DOF.
\end{description}

Note that all of the above depend on the  model used.

%\subsection{Prob. function as a collection of points with attached weight.}
%
%\subsection{Dynamic system}
%    It's a tuple <space,update,time>
%    Any mechanical system following newton physics (or relativistic for that matter) can be considered to be an dynamic system
%    in our case the tuple whould be <feature\_space,update\_rule (the one we are trying to preprocess data to approximate),time>



\chapter{Theory}
    \label{sec:theory}
    \section{Introduction} The core of the tracking engine is a technique
known as the \emph{particle filter}.  The particle filter is a kind of
Bayesian filtering where one uses discrete hypotheses, also known as
\emph{particles}, to approximate continuous PDF \cite{ProbRob}.  It
builds upon the theory of \emph{Markov processes} and the \emph{hidden
Markov model}.


\begin{figure} \centering
  \includegraphics[width=0.175\textwidth]{mp-graph.pdf}
  \caption{Schematic image of a Markov process.}
  \label{fig:hmm-graph}
\end{figure}


\section{Markov processes} A Markov process is a special case of a
stochastic process. For a Markov process, the next state depends only
on the present state and not on past states.  For this reason, a
Markov process is often said to be ``forgetful''.

In mathematical terms, a Markov process satisfies the following:
\begin{equation} p\left(Z_t|Z_{t-1} \wedge Z_{t-2} \wedge \dots \wedge
Z_0\right) = p\left(Z_t|Z_{t-1}\right),
\end{equation} $p\left(Z_t|Z_{t-1} \wedge Z_{t-2} \wedge \dots \wedge
Z_0\right)$ is the probability that the system will have state $Z_t$
at time $t$, given that the previous states where $Z_{t-1},
Z_{t-2},\dots, Z_0$.

\begin{figure} \centering
  \includegraphics[width=0.35\textwidth]{hmm-graph.pdf}
  \caption{Schematic image of a hidden Markov model.}
  \label{fig:hmm-graph}
\end{figure}

\section{The Hidden Markov Model}

The working principle of the particle filter is based on the
\emph{hidden Markov model} (HMM).  A HMM describes a Markov process
where we cannot measure the state directly - it is
``hidden''\cite{EncyclopediaMachineLearning}.  Instead we obtain an
\emph{observation} $I$\footnote{In this thesis, the observation is
always a grayscale \emph{image}, therefore the observation is denoted
$I$.}  of the state. This \emph{perception} is generally
non-deterministic, so we need to denote it as $p(I_t|Z_t)$ which is
the probability that we will observe $I_t$ if the state is $Z_t$.

\section{The Curse of Dimensionality} A phenomenon that becomes
apparent in high-dimensional spaces is the so-called ``Curse of
dimensionality'' \cite{EncyclopediaMachineLearning}.  The problem is
that the search volume grows exponentially with the number of
dimensions.  It originates from the fact that we need $\Ordo{C^n}$
samples to obtain a sample density of $C$ in a $n$-dimensional space.

The first consequence of this is that in order to approximate a
high-dimensional function one needs orders of magnitude more samples.

The other drawback with high dimensional space is the large
``borders'' of the sample-set compared to lower dimensional space
which results in orders of magnitude higher chance for an point one
want to approximate to fall outside the sample-set and needs to be
extrapolated instead of the better alternative of interpolation.

% For example, consider an interval on a number line, where we cover
% the middle third of the interval with samples.  If we randomly
% select a number from the interval, there is a one in three chance
% that our selected number is in the region with samples.  Take this
% to two dimensions, and we have covered only a ninth of the space,
% and so on. This means that the chance that the point we want to
% approximate falls outside the sample set is orders of magnitude
% higher in higher dimensional spaces, compared to lower dimensional
% ones.  We need more samples in order to be able to interpolate
% between samples.

\begin{example} Figure \ref{fig:curse-of-dimensionality} shows 128
randomly scattered points in 1, 2 and 3 dimensions. Notice how the
density decreases with increasing dimension.
  \begin{figure}
    \begin{tabular}{rcl}
      \includegraphics[scale=0.3,trim=4cm 4cm 4cm 4cm]{1D.pdf}&
      \includegraphics[scale=0.3,trim=4cm 4cm 4cm 4cm]{2D.pdf}&
      \includegraphics[scale=0.3,trim=4cm 4cm 4cm 4cm]{3D.pdf}
    \end{tabular}
    \caption{Plots of $128$ scattered samples in $1$, $2$ and $3$
dimensions, respectively.}
    \label{fig:curse-of-dimensionality}
  \end{figure}
\end{example}

\begin{example} For a $16$ DOF model one needs $10^{16}=10$
quadrillion datapoints to acquire a density of $10$ samples per unit
volume. Millions of gigabytes would be needed just to store the
samples.
\end{example}

\begin{example} In 2 dimensions it is sometimes feasible to use an
exhaustive search.  An example of this is the Hough transform
\cite{DigitalImageProcessing}, where the search is done through the
$\rho\theta$ space of line responses on images.
\end{example}

\subsection{Overcoming the Curse} One way to overcome the curse in the
context of tracking is to perform a directed search. Let the search be
in an $n$ dimensional space with a grid of $g$ grid lines in each
direction.

\begin{enumerate}
\item Use the information about the most recent\footnote{In the
    Bayesian case, the most recent estimate} location and assume that
  the tracked object cannot travel more than $R < g$ grid steps in one
  time step.  This reduces the volume of the (discrete) search space
  from $\Ordo{g^n}$ to $\Ordo{R^n}$.
      % \begin{proof}
      %   Change the coordinate system to generalized spherical
      %   coordinates $(r,\phi_1,\phi_2,...,\phi_{N-1})$ and fixate
      %   $r$ foreach $r<R$ which gives us a $(N-1)$ dimensional
      %   search surface, sum up the fixated $r$'s to get the total
      %   search space.
      % \end{proof}
\item With prior knowledge of how the tracked objects move
  \footnote{Such as the state transition probabilites $\cprobnext{Z}$
    in a HMM} we can direct our search to specific regions in the
  state space, depending on how probable it is for the tracked object
  to be located there. This reduces the size of the search space
  depending on how sure we are of the previous state.
\end{enumerate}

% This phenomen of having hue searchspace in highdimensinonal space is
% fairly common and have the name "Curse of dimensionality".

% One way to somewhat overcome this emptyness in space is to have a
% dynamic (active?) algorithm that adopts the sample density according
% to the models relative frequency in that area and this results in,
% for a given amount of samples, its more likely for an sampled model
% to occure in a more densly pre-sampled (pre-sampled?) area, in fact
% this method when doing it right (ideally) gives: given a set of
% samples the overall density for all the samples in the
% sampledatabase would be optimal) [proof for this will be given in
% <blabla>] ... (use the world hypotesis instead of sample, or perhaps
% sampled-hypotesis) (below is perhaps somewhat redudant, but one can
% pick bits and pieces from both) So having a bias towards trying out
% more plausible hypotesis for the models innerstate is better than
% doing a naive exhausted search tru the entire feature space, this
% could only be used if you have $\le$ 3 DOF as in for example finding
% straight edges with houghtransformation[ref].  ...  One method that
% uses prior knowledge of the models PDF and a pretrained database
% containing knowledge on how to approximate the models PDF in the
% next timestep is particle filter which is an (instance?) of the
% ideal bayesian filter.


% """ The "Curse of dimensionality", is a term coined by Bellman to
% describe the problem caused by the exponential increase in volume
% associated with adding extra dimensions to a (mathematical)
% space. One implication of the curse of dimensionality is that some
% methods for numerical solution of the Bellman equation require
% vastly more computer time when there are more state variables in the
% value function.

% For example, 100 evenly-spaced sample points suffice to sample a
% unit interval with no more than 0.01 distance between points; an
% equivalent sampling of a 10-dimensional unit hypercube with a
% lattice with a spacing of 0.01 between adjacent points would require
% 1020 sample points: thus, in some sense, the 10-dimensional
% hypercube can be said to be a factor of 1018 "larger" than the unit
% interval. (Adapted from an example by R. E. Bellman, see below.)
% """

% from: "R. Bellman, Adaptive control Processes, p.94, Princeton
% University Press, NJ, 1961."

% “In view of all that we have said in the foregoing sections, the
% many obstacles we appear to have surmounted. What casts the pall
% over our victory celebration? It is the curse of dimensionality, a
% malediction that has plagued the scientist from earliest days.”

% """
%
% Number of states grows exponentially in n (assuming fixed number of
% discretization levels per coordinate)
%
% """

%
% """ One solution on how to make the effects of the curse of
% dimensionality is to make a directed search..  """
%

%
% """ Bellmans dynamic programming (DP) requires knowledge of
% transition probablities of the dynamic system from ones state to the
% next """
%

\section{The Particle Filter}

One naive way to compute the state $Z_t$ would be to perform an
exhaustive search in the state space, and select the state for which
$\cprob{I_t}{Z_t}$ is maximised.<ref:COD>

The particle filter is a technique for reducing the search
space<ref:COD>. It uses a finite set $X_t$ of hypotheses to
approximate the PDF $\cprobnext{Z}$ of a HMM. The hypotheses $X_t$ are
also refered to as \emph{particles}, thereby the term ``particle
filter''.

\begin{figure}
  \centering
  \includegraphics[width=0.8\textwidth]{hmm-pf-graph.pdf}
  \caption{Schematic image of the particle filter alongside a HMM.}
  \label{fig:hmm-graph}
\end{figure}

Figure \ref{fig:hmm-graph} shows the principle of the particle filter
working alongside a hidden Markov model. The following is the core
function of the particle filter:

\begin{quote}
  \emph{The particle filter attempts to approximate the PDF
    $\cprobnext{Z}$ as a set $X_t$ of discrete hypotheses.}
\end{quote}

More particles mean greater accuracy, since the PDF can then be
approximated more closely. However, using many particles increses
computational cost.  Therefore the number of particles is an important
trade-off.  The particle filter employs a few tricks to \emph{filter}
the hypotheses, tending to keep probable ones and throwing improbable
ones away, in order to intelligently reduce the number of particles
needed for a good approximation. The filter works in four steps:

\begin{description}
\item[Prediction] The hypotheses $X_{t-1}$ are updated in the
  \emph{prediction} step to an approximation $\bar{X}_t$ of
  $\cprobnext{Z}$.  This is done by drawing new samples
  $\cprobnext{x}$, for each $x_{t-1}$ in $X_{t-1}$.
\item[Perception] By measuring the state of the system, we gain an
  \emph{observation} $I_t \sim \cprob{I_t}{Z_t}$ of the state $Z_t$.
\item[Filtering] The observation $I_t$ of the system is then used for
  filtering bad hypotheses out of $\bar{X}_t$.  We draw samples $X_t$
  from $\bar{X}_t$ with probabilities given by
  $\cprob{I_t}{\bar{X}_t}$. The result will be a surjection, where
  $X_t$ will be a subset of $\bar{X}_t$ where more probable hypotheses
  appear multiple times.<ref:problem without sigma> For this reason,
  this is also known as the \emph{resampling} step. The set $X_t$ is
  the \emph{belief}, our approximation of $\cprobnext{Z}$.
\item[Selection] Finally, we produce a single hypothesis $x_t$ from
  $X_t$ as our \emph{estimate} of the state $Z_t$. Assuming $X_t$ is a
  good approximation of $\cprobnext{Z}$, and that $\cprobnext{Z}$ is
  unimodal, the means value of $X_t$ is a good estimate since it
  approximates the expectation of $\cprobnext{Z}$.  If $\cprobnext{Z}$
  is multimodal, however, the mean could be a bad estimate since the
  expectation may be very improbable.
\end{description}

The reason why this works is the following theorem, which is Theorem
3.1 in \cite{Hedvig} with some modifications to notation:

\begin{theorem}
  Given the PDFs $\cprobnext{x}$, $\cprob{I_t}{x_t}$ and
  $\cprob{x_{t-1}}{I_{t-1} \wedge \dots \wedge I_0}$, the PDF
  $\cprob{x_t}{I_t \wedge \dots \wedge I_0}$ can be expressed as

\begin{equation}
  \cprob{x_t}{I_t \wedge \dots \wedge I_0} = \kappa \cprob{I_t}{x_t} \int{ \cprobnext{x}\cprob{x_{t-1}}{I_{t-1} \wedge \dots \wedge I_0} \mathrm{d}x_{t-1}},
  \label{thm:pf-grand}
\end{equation}
where $\kappa$ is a normalization constant

For the proof of this, see \cite{Hedvig}.

\end{theorem}

Let's sketch out the connection between this and what we've done so
far. For a more rigorous derivation, see chapter 4 of \cite{ProbRob}.

In this thesis, $\cprob{x_t}{I_t \wedge \dots \wedge I_0}$ is
approximated as the set $X_t$ of hypotheses. Substituting this into
\eqref{thm:pf-grand} and replacing the integral with a sum yields

\begin{equation}
  X_t = \kappa \cprob{I_t}{x_t} \sum { \cprobnext{x} X_{t-1} }.
\end{equation}

Note that this is horrible abuse of notation, but from here we can
take the step to

\begin{equation}
  X_t \sim \cprob{I_t}{x_t} \cprobnext{x}
\end{equation}

\subsection{The Particle Filter algorithm}
\begin{table}
  \begin{codebox}
    \Procname{$\proc{Particle-Filter} (X_{t-1},I_t)$}
    \li $\bar{X}_t \gets \emptyset$
    \li \ForEach $x_{t-1} \in X_{t-1}$
    \li \Do
    \li $x_t \gets \proc{Predict}(x_{t-1})$
    \li $w \gets \proc{Importance}(x_t,I_t)$
    \li Append $\left<x_t, w\right>$ to $\bar{X}_t$
    \End
    \li
    
    \li $X_t \gets \emptyset$
    
    \li \While $\abs{X_t} < \abs{\bar{X}_t}$
    \li \Do
    \li Take $\left<x_t, w\right>$ from $\bar{X}_t$ with probability $\propto w$
    \li Append $x_t$ to $X_t$
    \End
    \li \Return $X_t$
  \end{codebox}
  \caption{The particle filter algorithm.}
  \label{alg:pf}
\end{table}

Table \ref{alg:pf} shows the particle filter algorithm. Note that the
functions \textsc{Predict} and \textsc{Importance} are unspecified -
they are problem specific. They correspond to the PDF $\cprobnext{x}$
and $\cprob{I_t}{x_t}$, respectively.

% We draw a set $\bar{X}_t = \xtmN{x}{t}{i}{N}$ of samples from
% $p$. These samples roughly represent a PDF for the current state
% $x_t$, but we have yet to consider our observation.  Therefore, we
% will create a new PDF weighted by how probable the observation $z_t$
% is. For each $x_t^m$, we let $w_t^m := q\left(z_t |
%   x_t\right)$. This defines a discrete PDF where $x_t^m$ is assumed
% with a probability proportional to $w_t^m$. From this final
% distribution we again draw $N$ samples $X_t$, which will be our
% estimate of the current state.  The elements $x_t^m$ are referred to
% as \emph{particles} and the set $X_t$ as the \emph{belief at time
%   $t$}.

In this thesis, the real $x_{t-1}$ is not known. Rather $x_{t-1}$ is
estimated with a set $X_{t-1}$ of $N$ particles.

\section{Visual Cues}
\begin{quote}
    The biggest problem with computer vision is that computers do not have
    vision, only a data input device in the form of a camera.
\end{quote}
A \emph{visual cue} is an image transformation $\phi$ that extracts
some property of the image, such as intensity
\footnote{In our case this is possible since we have an homogenous object
in the form of a backlit rodent}, edges, ridges\cite{Hedvig} or different 
refinements as in section \ref{prep-real}.


\begin{theorem} %TODO svagare?
  \label{thm:response_max}
  Let $f$ be a positive Riemann function with compact support,
  defined on a set $\Omega$. Then
  \begin{equation}
    \argmax{\bar{e}}
    \left( \int\limits_{\Omega}{f(\bar{x})f(\bar{x}-\bar{e})} \right)
    =0
    \label{thm:response-max-eqn}
  \end{equation}
\end{theorem}
\begin{proof}
  Firstly define the window function as

  \begin{IEEEeqnarray*}{s"r?c?l}
    & W_a^b(x) & = &
    \begin{cases}
      0,~& x<a\\
      1,~& a \leq x \leq b\\
      0,~& x>b
    \end{cases}\\

    Multiplication: &
    (W_a^bW_c^d)(x) & = & W_{\max(a,c)}^{\min(b,d)}(x)\\

    Translation: &
    W_a^b(x-e) & = & W_{a+e}^{b+e}(x)\\

    Integration: &
    \int\limits_{\RR}{W_a^b(x)dx} & = & \Theta(b-a).
  \end{IEEEeqnarray*}
  
  \begin{IEEEeqnarray*}{L?c?l}
    \argmax{e}\left( \int{W_a^b(x)W_a^b(x-e)} \right) &=& \\
    \argmax{e}\left( \int{W_a^b(x)W_{a+e}^{b+e}(x)} \right) &=&\\
    \argmax{e}\left( \int{W_{\max(a,a+e)}^{\min(b,b+e)}(x)} \right) &=&\\
    \argmax{e}\left( \Theta(\min(b,b+e)-\max(a,a+e)) \right) & = & 0.
  \end{IEEEeqnarray*}
  
  This trivially holds for superposition of windows, since all windows
  will scale and translate the same way.  With a finite support, $e=0$
  is the only solution. Additionaly, this also holds in higher finite
  dimensions since we can just repeat the process one dimension at a
  time.
    
  All riemann functions can be written as a superposition of windows
  like this
  \begin{equation*}
    f(x)=\sum{c_iW_{a_i}^{b_i}(x)},
  \end{equation*}
  
  and therefore \eqref{thm:response-max-eqn} holds for any Riemann function $f$ with finite support.
  
\end{proof}

\section{Model}

%"
%==Model selection
%Model selection is the process of choosing an appropriate mathematical model
%from a class of models.
%" - Encyclopedia 

A whisker can be modeled as a function:

% In the following chapter we will formalize the whisker and discuss
% our model selection\cite{EncylopediaMachineLearning} process.

% First of we formalize the whisker by the definition

\begin{definition}
    Let 
    \begin{equation}
    \begin{split}
        \text{whisker} : \RR^+ &\rightarrow \RR^2 \\
                  \omega&\mapsto \text{whisker}(\omega)
    \end{split}
    \end{equation}
\end{definition}
where $\omega$ is a coordinate along the length of the whisker and
$\text{whisker}(\omega)$ is a point in the $x$-$y$ plane.

The tracking problem is then to find a function $\text{whisker}^*$
that approximates $\text{whisker}$. The function class used in a
model must satisfy the following conditions:

\begin{enumerate}
\item It must be able to approximate the whisker sufficiently
  well.

  \begin{example}
    A straight line will not suffice since the $\text{whisker}$ is
    generally curved and straight lines can not fill that partition of
    the space.
  \end{example}
  
\item The functions must be $C_1$ and be possible to represent with a
  finite number of parameters.
\end{enumerate}

With this in mind, two classes of functions immediately appear as
candidates:
\begin{itemize}
\item Polynomials $\sum_{i=0}^{n} a_i\omega^i$
\item Fourier series $\sum_{k=0}^{n} a_k\sin(\frac{k\pi\omega}{L}) +
  b_k\cos(\frac{k\pi\omega}{L})$
\end{itemize}

A brief analysis of both will follow, after a discussion of
simplifications and difference measures..

\subsection{Simplifications}
One simplification one might make is to assume that the root of the
whisker is fixed in some point on the snout. This means that we can
let the whisker function be defined in a head-fixed coordinate system
for each whisker, with the root of the whisker at the origin. This
gives us the boundary condition
\begin{equation}
    \label{eq:bv_root}
    \text{whisker}^*(0)=\bar{0}.
\end{equation}

Manual inspection of whisker videos suggests that this is not the case
for real whiskers. However, there seems to be some point within the
snouts that stays approximately still and can be regarded as the root
of the whisker. A better model could take this into account, but that
will not be covered in this thesis.

The thickness of a whisker is not constant, but decreases as the
distance from the head increases. A simple model for this is to define
the whisker thickness as

\begin{equation}
    d(\omega) = \begin{cases}
        D-\frac{D\omega}{L},~& \omega<L\\
        0,~& \omega\ge L
    \end{cases}
\end{equation}

where $D>0$ is the thickness at the root and $L>0$ is the total length of the
whisker.

\subsection{Difference measure}
The standard way to quantify distance between functions defined on an
interval $\interval{a}{b}$ is to use the norm in the
$\Lp[2](\interval{a}{b}, 1)$ Hilbert space. In this thesis, the norms
for $p = 2, 4, 8$ will be used, and the impact of the choice of $p$
will be investigated. This means that condition 1 above says the model
must be such that the $\Lp$ distance $\Lpnorm{\text{whisker} -
  \text{whisker*}}$ can be made sufficiently small.

\subsection{Polynomial $\Spline{\omega}$}
The first and simplest candidate is the polynomials. Manual tests in
MATLAB indicate that three terms are enough to approximate a whisker
well enough that the difference is not visible to the naked eye. A
whisker function can therefore be modeled as a third degree polynomial
$\Spline{\omega}$, also known as a \emph{spline}.

These define parameterized curves in the $xy$ plane as
\begin{equation}
  (\omega,\Spline{\omega})
\end{equation}

This choice can be justified by comparing with the theory of beams
under small deformations in solid mechanics. After all, a whisker is
not too different from a beam. The two main assumptions for this to
hold is that deformations are small and that the effects of motion on
whisker deformation are negligible. \cite{Hallfasthet} This may not
quite be the case, but a spline is still capable of approximating the
momentaneous shape of a whisker well enough.

\subsection{Sine series $\sum a_k\sin \left(\frac{k\pi\omega}{L}\right)$}
Another promising candidate is the Fourier series. Considering
\eqref{eq:bv_root}, the Fourier series is reduced to a sine series
$\sum a_k\sin \left(\frac{k\pi\omega}{L}\right)$. However, such a
series will always be zero at $\omega = L$, and is therefore not a
reasonable model for whiskers. To address this, one could instead use
quarter-periods: $\sum a_k \sin\left( \frac{k\pi\omega}{2L}\right)$,
but at the cost of losing the orthogonality property of the sine
basis. Manual tests in MATLAB indicate that such a series to third
order performs approximately the same as a spline.

These define parameterized curves in the $xy$ plane as
\begin{equation}
    (\omega,\sum{a_n\sin (\frac{2\pi n}{L}\omega)})
\end{equation}


The choice of a sine series can be justified by comparing with the
theory of stiff strings in analytical mechanics. The movement equation
for a stiff string is a partial differential equation containing the
second and fourth spatial derivatives.\footnote{Source: Solving
  exercise 7.10 of \cite{VarKalk} using variational calculus.} A
classic separation of variables solution to the equation would be a
sine series for the spatial part.

\subsection{Theoretical evaluation}

It is hard to theoretically justify the choice of $\text{whisker}^*$
model since we do not have an analytic model for the $\text{whisker}$.

\begin{figure}
  \centering
  \includegraphics[width=0.45\textwidth]{rat-splines.png}
  \includegraphics[width=0.45\textwidth]{rat-sines.png}
  \caption{Comparison of whisker models. Left: Splines, Right: Sine series}
  \label{fig:model-comparison}
\end{figure}

For this thesis, the spline model was used in the tracking engine. The
main reason for this is that it is slightly simpler than the sine
model. It also is rather easy to get an intuitive grasp of how the
parameters affect the curve shape - the $\omega^3$ term mostly affects
the tip of the whisker while the $\omega$ term affects the overall
orientation.



\chapter{Algorithms and Implementations}
    \label{sec:algorithms_implementations}
    The testing implementation was developed with high modularity in mind,
since it is meant to be a proof-of-concept implementation and not a
production grade system. High modularity also makes development easier
and the system more robust against changes, two very important
qualities during this project.

The implementation is written in Python using NumPy and SQLite3, and
consists of three main parts:
\begin{description}
  \item[Particle Filter] A direct  implementation of the procedure in
    table \ref{alg:pf}.
  \item[Database] A database with functions for extracting transition
    hypotheses. Provides the prediction PDF $\cprobnext{x}$ to the
    particle filter.
  \item[Tracker] Manages the model and performs matching between
    hypotheses and images. Provides the filtering PDF
    $\cprob{I_n}{x_n}$ to the particle filter.
\end{description}

\section{The particle filter}
The particle filter implementation is a direct implementation of the
procedure in table \ref{alg:pf}. It is implemented as a function that
takes the parameters $X_{t-1}$, $I_t$, \texttt{importance\_function} and
\texttt{sampling\_function}. The parameters are the hypotheses from
the last time step, the current video frame and the functions to use
as \textsc{Predict} and \textsc{Importance} in \ref{alg:pf},
respectively. This means that the particle filter function is general
and independent of the model used. The implementations of
\textsc{Predict} and \textsc{Importance} are provided by the database
and tracker, respectively.

\subsection{Initilization $x_0$}
The test implementation needs to be manually initialized. When
tracking generated whiskers, the states were always known and the
initialization could therefore be programmatically inserted. When
testing on real whiskers, the start states were calculated by manually
selecting five or six pixels along each whisker and using a MATLAB
script to find the least squares solution for the coefficients
$\coeffs$. The problem of automatic
initialization is a difficult one \cite{Hedvig}, and is not covered in
this thesis.

\section{The state transition database}

\subsection{Data format}
A \emph{state transition} is a pair $\trans$ that denotes we have
observed a system go from state $\tf$ to state $\tt$ in one time
step. Technically, the state transition database is implemented as an
SQLite3 database. One transition is represented in the database as a
row with the state parameters of the model before and after the
transition. The set of transitions in the database will be denoted \TS.

\subsection{Prediction $\cprobnext{x}$}

\begin{table}[h]
  \begin{codebox}
    \Procname{$\proc{DB-Predict} (x_{t-1})$}
    \li $ x_t \gets 0$
    \li $ W \gets 0$
    \li \ForEach $(\tf, \tt) \in \TS$
    \li \Do
      \li $ w \gets \left(\Lpnorm[p]{\tf - x_t}\right)^{-a}$
      \li $ x_t \gets x_t + w \cdot \tt$
      \li $ W \gets W + w$
    \End
    \li Take $v \sim \ndist{0}{\Sigma}$
    \li \Return $ x_t / W + v $
  \end{codebox}
  \caption{Pseudocode for the prediction function. Notice the parameters $a$ and $p$.}
  \label{alg:predict}
\end{table}

The \textsc{Predict} function in table \ref{alg:pf} is implemented as
a weighted mean of the state transitions in the database. The function
is stated in table \ref{alg:predict}. Notice the parameters $a$ and
$p$. $p$ is a positive integer that determines which $\Lp$ space to
compute the norm in. $a$ is a positive number, and determines how fast
the weight $w$ declines with the distance $\Lpnorm[p]{\tf -
  x_{t-1}}$. A high $a$ means closer transitions get a much higher
weight than ones far away, see figure \ref{fig:x-to-the-minus-a}.

At this point, however, the prediction is still deterministic - the
result for any given input $x_{t-1}$ is completely determined by the
parameters $a$ and $p$ and the content of the database. A
deterministic prediction function is not desirable, since the
filtering step only removes improbable hypotheses and replaces them
with duplicates of probable ones. This means that having a
deterministic prediction effectively reduces the number of hypotheses
with each filtering step. For this reason, the result is offset by a
small normal distributed term\footnote{The offset $v$ is a polynomial
  $\Spline[b]{\omega}$ where coefficient $b_i \sim \ndist{0}{\sigma_i}
  $ and $\sigma_i$ is different for the different $i$.} $v$ to make
the prediction nondeterministic.

\begin{figure}[ht]
  \centering
  \includegraphics[width=0.4\textwidth,trim=65mm 65mm 65mm 65mm]{x-to-the-minus-a.pdf}
  \caption{Transitions closer to $x_{t-1}$ recieve a much greater
    weight if $a$ is large.}
  \label{fig:x-to-the-minus-a}
\end{figure}

\section{Tracker}

The tracker uses the whisker model described in chapter 4 and
internally represents whiskers with the tuple $\coeffs$ of polynomial
coefficients.

\subsection{Filtering $\cprob{I_t}{x_t}$}
\label{sec:filtering}
The \textsc{Importance} function in table \ref{alg:pf} is implemented
simply as the response of $x_t$ on $I_t$, raised to a power $g$. A
high $g$ means that the peak in the response is further
amplified. Figure \ref{fig:example-mask} shows an example of a
rendered hypothesis, as used for computing the response. See section
\ref{sec:response} for details on the response.

\begin{table}[h]
  \begin{codebox}
    \Procname{$\proc{Importance} (x_t, I_t)$}
    \li \Return $\Response{x_t}{I_t}{\phi}^g$
  \end{codebox}
  \caption{Pseudocode for the importance function. Notice the parameter $g$.}
  \label{alg:importance}
\end{table}

The \textsc{Importance} function in table \ref{alg:pf} as being the response for $x_t$ on $I_t$.
\begin{figure}[h]
  \centering
  \includegraphics[width=0.38\textwidth]{example-mask.png}
  \caption{Example rendered image $R(x_t)$ for some hypothesis $x_t$.}
  \label{fig:example-mask}
\end{figure}

To indicate how the response works for generated and realdata we look at the following figures.
Figure \ref{fig:response_generated} shows the response curve for a generated image with parameters $(0,0,0)$ when varying the different parameters in the polynomial model.
Figure \ref{fig:response_real} shows the response curve for real images which has approximatly the parameters $(0,0,0)$, we can clearly see that the curve is far from perfect. 
The ground truth peak is still around $(0,0,0)$ but much less appearent.
The faulty peeks is generated by clutter from the other whiskers and we can see a tendency for higher values for positive $a_2,a_3$ which is probably due to motion blur effects.

\begin{figure}
    \centering
    \includegraphics[width=\textwidth]{response_curves.pdf}
    \caption{
        Response curves over the parameters $(a_3,a_2,a_1)$ of the polynomial model on a generated whiskers image with the parameters $(0,0,0)$. With $\phi=1$.
    }
    \label{fig:response_generated}
\end{figure}
\begin{figure}
    \centering
    \includegraphics[width=\textwidth]{response_curves_real.pdf}
    \caption{
        Response curves over the parameters $(a_3,a_2,a_1)$ of the polynomial model on a real whisker image, $\prob{whisker}$ from \ref{sec:findwhisker},
        with approximatly the parameters $(0,0,0)$. With $\phi=1$.
    }
    \label{fig:response_real}
\end{figure}

\section{Preprocessing of real images}
\label{prep-real}

In the images of real rodents used for testing, the rodent was
illuminated from below, meaning it and its whiskers were dark against
a bright background as in the figure \ref{fig:original_bg}. The filtering function described in the previous
section expects whisker pixels to be bright, so the images had to be
inverted. However, the body had the same pixel values as the whiskers,
meaning no conclusions can be drawn simply by inspecting the value of
a pixel. Therefore the following steps also had to be taken.

\subsection{Background subtraction}
Removing effects like difference in background illumination and static
objects is preferred in order to minimize faulty responses.

Assuming we always have a static camera setup for each sequence and
that we have an adequate sequence without the rodent, the background
$BG$ was captured a priori by taking the mean $I_{BG}$ of the first
100 frames as seen to the left in figure \ref{fig:orignal_bg}.\footnote{A sufficient number} Then we can simply subtract
the background $I_{BG}$ from the orignal image $I$ and getting the result shown to the right in figure \ref{fig:subtract}: \footnote{Worth
noting is that more sophisticated background subtraction methods
exist, but this solution works relatively well considering the
simplicity.}

\begin{equation}
  I_{FG} = I - I_{BG}.
\end{equation}

\begin{figure}
\begin{center}
    \includegraphics[width=0.45\textwidth]{frame-0759.png}
    \includegraphics[width=0.45\textwidth]{frame-0759_bg.png}
\end{center}
\caption{The image to the left shows the original image, to the right we have the mean static background.}
\label{fig:orignal_bg}
\end{figure}
\begin{figure}
\begin{center}
    \includegraphics[width=\textwidth]{frame-0759_sub.png}
\end{center}
\caption{The image shows the effect of subtracting the static background.}
\label{fig:subtract}
\end{figure}

\subsection{Extract the body $\prob{\text{body}}$}
\label{sec:findbody}
Extracting the body has two purposes. First, to find the head-fixed
coordinate system. Second, to subtract the body from the image in
order to highlight the whiskers.

\begin{definition}
  The image
  \begin{equation}
    \prob{\text{body}}\in \IS
  \end{equation} is an estimation if the probablity of the pixel
  being a body pixel.\footnote{The estimate only needs to
    be an indication of whether it is a body}
\end{definition}

We will assume that the only non-static objects in the image are the
whiskers and body. One simple feature that easily classifies
$\prob{\text{whisker}}$ from $\prob{\text{body}}$ is size. Blurring
the image removes fine structures and retains larger which can be seen to the left in figure \ref{fig:blured_snout}.  This was
performed by convoluting the image with a Gaussian with
${\sigma=9}$px:\footnote{The value of $\sigma$ was obtained by manual
  testing and inspection.}

\begin{equation}
  \prob{\text{body}} = I_{blur} = I_{FG} \star \ndist{0}{\sigma}.
\end{equation}
Note that $\ndist{\mu}{\sigma}$ traditionally denotes a set of normal
distributed stochastic variables. In this thesis, we will also use
this to denote the PDF of the normal distribution, and let the context
provide the distinction.

$\prob{\text{body}}$ was then used to create a \emph{mask} $I_{\text{body}}$:
\begin{equation}
  I_{\text{body}} = \prob{\text{body}}>0.6 \in \IS,
\end{equation}
meaning $I_{\text{body}}$ is white where $\prob{\text{body}}$ is greater than
0.6 and black everywhere else, the result is displayed to the right in figure \ref{fig:blured_snout}.\footnote{The 0.6 threshold was found,
through manual testing, to make the body stable between frames.}

\begin{figure}
\begin{center}
    \includegraphics[width=0.45\textwidth]{frame-0759_blured.png}
    \includegraphics[width=0.45\textwidth]{frame-0759_snout.png}
\end{center}
\caption{The image to the left shows the blured image with ${\sigma=9}$px, to the right we have the snout mask.}
\label{fig:blured_snout}
\end{figure}

\subsection{Extract the whiskers $\prob{\text{whisker}}$}
\label{sec:findwhisker}
The filtering described in section \ref{sec:filtering} needs an
indicator for the locations of whiskers.

\begin{definition}
  The image
  \begin{equation}
    \prob{\text{whisker}}\in \IS
  \end{equation} is an estimation of the probablity of a pixel being
  a whisker pixel.\footnote{The estimate only needs to be an
    indication of whether it is a whisker.}
\end{definition}

This is naively performed by masking with the inverse
$\bar{I}_{\text{body}}$ of $I_{\text{body}}$:

\begin{equation}
  \prob{\text{whisker}} = I_{FG} * \bar{I}_{\text{body}}
\end{equation}
which finally gives the results shown in figure \ref{fig:whiskers}.

\begin{figure}
\begin{center}
    \includegraphics[width=\textwidth]{frame-0759_whiskers.png}
\end{center}
\caption{The finished preprocessed image.}
\label{fig:whiskers}
\end{figure}

\subsection{Find the snout coordinate system}
The whisker model does not take head movements into account, and
therefore expects whisker roots to be stationary throughout the
sequence. Therefore, the video is translated such that the head stays
approximately fixed throughout the image sequence.

For this, we use a transformation $\phi$ that extracts the shape of
the body. $\phi$ first blurs the image\footnote{The value $\sigma = $
  5 pixels was obtained through manual testing.}, to smooth out the
response, then applies a Prewitt filter:
\begin{IEEEeqnarray*}{rCl}
  \phi_{\text{blur}}(I) &=& I \star \ndist{0}{5\mathrm{px}}\\
  \phi(I) &=& \sqrt{(\phi_{\text{blur}}(I) \star \text{Prewitt}_x)^2 +
    ( \phi_{\text{blur}}(I) \star \text{Prewitt}_y )^2}.
\end{IEEEeqnarray*}
that extracts the shape of the body.

The first frame where the snout is fully visible is hand picked
and used to create an image $I_{\text{ref}} = \phi(I_{\text{body}})$,
where $I_{\text{body}}$ is created through the steps detailed in
section \ref{sec:findbody}.

A local search within $5$px from the last location is then
performed\footnote{This could easily be extended to include rotation
  as well.}  for each body image $I_{\text{body}}$ in the
sequence. The translation $(\Delta x, \Delta y)$ from $I_{\text{ref}}$
is extracted:\footnote{In the same way as $\Response{\cdot}{\cdot}{\phi}$ but without the hypothesis.}
\begin{equation}
  (\Delta x, \Delta y) = \argmax{(x,y)~\text{close}}
  (\sum 
  \text{translate}(I_{\text{ref}},-(x,y))*\phi(I_{\text{body}})
  )
  \in \ZZ^2,
\end{equation}

and the frame is then translated accordingly.



%TODO results or Results?
%TODO does one write benchmarking? and results?
\chapter{Results}
    \label{sec:benchmarks_results}
    The results consist of three main parts:
\begin{itemize}
\item A benchmark of the impact of different parameters on the
  tracking performance, performed on synthetic videos and evaluated
  programmatically
\item Test runs on real data with the best performing parameters from
  the benchmark, evaluated manually
\item Conclusions concerning the feasibility of a probabilistic
  whisker tracking system.
\end{itemize}

\section{Parameter Benchmark}

A benchmark was performed in order to identify how tracking
performance is affected by the different parameters. The benchmark was
performed on generated videos of synthetic whiskers, which enabled
programmatic evaluation of the results since the correct shapes of the
whiskers were known.

\subsection{Test data}
\label{sec:test-data}

The test data consisted of a single generated video of synthetic
whiskers. The video contained 6 whiskers and was 64 frames long. Each
whisker had length 200 and a random base shape $\Spline{\omega}$, with
$a_3, a_2, a_1$. Each $a_i$ was in the range $\left[0,
  \sigma_i\right)$, where $\sigma_3 = 1.6 \cdot 10^{-5}, \sigma_2 =
4\cdot 10^{-3}, \sigma_1 = 1$. Each whisker was then assigned a random
phase $d \in \left[0, 2\pi\right)$, and the shape at time step $t$ was
$\left(\Spline{\omega}\right) \sin(\frac{2\pi t}{30} + d)$. The
frequency $\frac{1}{30}$ was selected through manual inspection of a
video of real whiskers. The resulting whiskers were roughly
reminiscent of real whiskers, and 6 sample frames can be seen in
figure \ref{fig:benchmark-video}.

The database was generated with the same parameters, and contained
$2^{14} = 16384$ transitions. Each transition consisted of a ``from''
part $\tf$ and a ``to'' part $\tt$. $\tf$ was created by generating a
base state and phase in the same way as for the video, and setting
$t=0$. $\tt$ was created by taking $\tf$ and increasing the phase by
$\frac{2\pi}{30}$.

\begin{figure}
  \centering
  \begin{tabular}{ccc}
    \includegraphics[width=0.3\textwidth]{benchmark-video/frame-00000.png} &
    \includegraphics[width=0.3\textwidth]{benchmark-video/frame-00010.png} &
    \includegraphics[width=0.3\textwidth]{benchmark-video/frame-00020.png}\\
    Frame 0 & Frame 10 & Frame 20\\
    \includegraphics[width=0.3\textwidth]{benchmark-video/frame-00030.png} &
    \includegraphics[width=0.3\textwidth]{benchmark-video/frame-00040.png} &
    \includegraphics[width=0.3\textwidth]{benchmark-video/frame-00050.png}\\
    Frame 30 & Frame 40 & Frame 50
  \end{tabular}
  \caption{Sample frames from the testing video.}
  \label{fig:benchmark-video}
\end{figure}  


\subsection{Evaluated parameters}
The following parameters were evaluated:

\begin{description}
\item[n] Number of particles
\item[p] In which $\Lp$ space to compute $\Lpnorm{\tf - x_{t-1}}$ in
  the prediction step
\item[a] The exponent for the weights in the prediction step, $w =
  \left(\Lpnorm{\tf - x_{t-1}}\right)^{-a}$
\item[$\sigma$] Standard deviation modifier for the offset in the
  prediction step. Standard deviations for the $\omega^3, \omega^2$
  and $\omega$ terms are $0.1\sigma\sigma_3, 0.1\sigma\sigma_2$ and
  $0.1\sigma\sigma_1$, respectively.\footnote{See section
    \ref{sec:test-data} for the $\sigma_i$ values.}
\item[g] The exponent for the importance in the filtering step
\end{description}

This yields the evaluation tensor $\Phi_{n,p,a,\sigma,g}$.

TODO (check the comments here)
%TODO define phi:img->img
%TODO define \phi(R(x))*\phi(I) as <x,I>_\phi

The measures that will be used as fitness of the parameters:
    \begin{description}
        \item[$\int{||\epsilon(t)||_{L^p}}dt$]
            integrating over time the the difference
            with the ground truth (do this for L{1:10} and see if it correlates with
            the p choosed (to see how much it deviates from the ground truth)
        \item[$\int{\Response{x_t}{I_t}{\phi} }dt$] 
            integrating over time the response
            for the choosed hypothesis (to see how the different image transformation
            affects the results, that is if it only follows what it thinks is best
            (phi))
        \item[Subjective] 4 image samples
    \end{description}
And all this are done for all 4 benchmark videos.


"There are many metrics by which a model may be assessed." - Encyclopedia


The fitness test was runned on different machines but this wont effect the
result since we initially wont consider the running time.

The runtime for the algorithm is handled separately on one machine setup <...>



Tillvägagångsätt:

1. Since we have prior knowledge about the effect off varying the parameters n,N
they will firstly be set to a sufficently large value.
2. A partion of the test-matrix will then be evaluated by ... parameter group 


\chapter{Analysis and Discussion}
    \label{sec:analysis_discussion}
    
Care must be taken when doing this type of experimental analysis on generated whiskers, and then
applying the conclusions on real whiskers. 
To quote Encyclopedia of Machine Learning\cite{EncyclopediaMachineLearning} under the section ``Algorithm Evaluation''
\begin{quote}
    However, much machine learning
    research includes experimental studies in which algorithms 
    are compared using a set of data sets with little
    or no consideration given to what class of applications
    those data sets might represent. It is dangerous to draw
    general conclusions about relative performance on any
    application from relative performance on this sample
    of some unknown class of applications. Such experimental
    evaluation has become known disparagingly as a bake-off.
\end{quote}

But the results should still probably give some direction for the parameters for further studies on real whiskers.




%Note on data preperation <encylcopedia>
%on the database training:
%"much of the theory on which learning systems are based assumes that the training data are a random sample of 
%the population about which the user wishes to learn a model.  Howe, much historical data represent biased 
%samples, for example, data that have been easy to collect or thath have been considered interesting for some other purpose."
%Our data is generated randomly and would work as perfect as can be BUT
%the bias will have a great effect when running on realdata... mention timescale problems for example, (worked like shit) 
%


%use the word dataset, training data, test data(generated) test data(real) [test set] mention bias

%analysis on the use of analytical solution for Lp which is norm on the function itself (seen as an inf-dimensional vector) instead of its parameters (which we tested and had som problems with)


%supervised learning (add in text and analysis)



%
%
%
%
%
%
%

%
% 
%
%
%
%
%

%One cant just see the tracking by just trying one step at the time in the same way as the database "generated" since we must measure the tendency for the
%algorithm to break down after a while, althoug it could be a fast interesting measure that we easily could have runned to see short term effects of 
%changing the parameters then going up to longer term effects.



%"A learning algorithm must interpolate appropriate predictions for regions of
%the instance space that are not included in the training data." - Encyclopedia
%(>model evaluation)


%Can a particlefilter overfit?(should this perhaps be in theory?)

    \label{sec:future}
    \input{future}

\input{bibliography}

%\input{appendixA} howto?
%\appendix
%\addappheadtotoc

%\chapter{RDF}\label{appA}


%\begin{figure}[ht]
%\begin{center}
% And here is a figure
% \caption{\small{Several statements describing the same resource.}}\label{RDF_4}
% \end{center}
% \end{figure}
%
%that we refer to here: \ref{RDF_4}


\end{document}
