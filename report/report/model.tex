\section{Model}

%"
%==Model selection
%Model selection is the process of choosing an appropriate mathematical model
%from a class of models.
%" - Encyclopedia 

In the following chapter we will formalize the whisker and discuss our model
selection<ref:encyclo> process.

First of we formalize the whisker by the definition

\begin{definition}
    Let 
    \begin{equation}
    \begin{split}
        whisker : \RR^+ &\rightarrow \RR^2 \\
                  \omega&\mapsto (x,y)
    \end{split}
    \end{equation}
\end{definition}

The idea is to find a function $whisker^*$ that approximates the $whisker$. The 
following class of functions are available for consideration:

It must be able to sufficiently approximate the whisker
\begin{equation}
    \forall whisker \exists whisker^* : whisker \approx whisker^*
\end{equation}
that is, the model must be able to fill the range/image of the $whisker$ function.

\begin{example}
    A straight line will not suffice since the $whisker$ is generally
    curved and straight lines can not fill that partition of the space.
\end{example}

Additionally it must of course be $C_1$ and have finitely many parameters.

We might also assume that a whisker has one fix point at the origin, which we
choose as origo, the boundary condition is equivalent to
\begin{equation}
    \label{eq:bv_root}
    whisker^*(0)=\bar{0}
\end{equation}
this is not the case in the real data, generally we have time dependency
\begin{equation}
    whisker(0,t)\neq\bar{0} 
\end{equation}
since rodents seems to have the ability to move around their whiskers root on
the chins surface.

We are for all cases assuming that the whisker thickness is defined by

\begin{equation}
    d(\omega) = \begin{cases}
        D,~& \omega=0\\
        D-\frac{D\omega}{L},~& \omega<L\\
        0,~& \omega\ge L
    \end{cases}
\end{equation}

where $D>0$ is the thickness at the root and $L<0$ is the total length of the
whisker.

\subsection{Polynom $\Spline{\omega}$}

    Our first and simplest candidate is the third degree polynom $\Spline{\omega}$.

    Parametrically define as
    \begin{equation}
        (\omega,\Spline{\omega})
    \end{equation}

    This chose can be justified by the theory of beams under small deformations 
    in the area of strength of materials, after all a whisker is not to far 
    away from a beamer.<ref: strength of materials> The two main assumption
    for this to hold is a linear elastic material and small deformations which
    ...

    <example images compared with real whiskers>

\subsection{Sinus serie$\sum{a_n\sin (\frac{2\pi n}{L}\omega)}$}
    Another promising candidate is the sinus series.
    The Fourier transform becomes $\sum{a_n\sin (\frac{2\pi n}{L}\omega)}$ when
    considering (\ref{eq:bv_root}). One can choose just a few $n$ which can be
    viewed as a few different waves propagating trough the whisker propelled by
    the whisking. 

\subsection{Verdict(?)}
    It is hard to analytically justify the chose of $whisker^*$ since we do not
    have an analytic model for the $whisker$.




