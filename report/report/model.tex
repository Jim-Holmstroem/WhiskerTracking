\section{Model}

%"
%==Model selection
%Model selection is the process of choosing an appropriate mathematical model
%from a class of models.
%" - Encyclopedia 

A whisker can be modeled as a function:

% In the following chapter we will formalize the whisker and discuss
% our model selection\cite{EncylopediaMachineLearning} process.

% First of we formalize the whisker by the definition

\begin{definition}
    Let 
    \begin{equation}
    \begin{split}
        \text{whisker} : \RR^+ &\rightarrow \RR^2 \\
                  \omega&\mapsto \text{whisker}(\omega)
    \end{split}
    \end{equation}
\end{definition}

The tracking problem is then to find a function $\text{whisker}^*$
that approximates $\text{whisker}$. The function class used in a
model must satisfy the following conditions:

\begin{enumerate}
\item It must be able to approximate the whisker sufficiently
  well.

  \begin{example}
    A straight line will not suffice since the $\text{whisker}$ is
    generally curved and straight lines can not fill that partition of
    the space.
  \end{example}
  
\item The functions must be $C_1$ and be possible to represent with a
  finite number of parameters.
\end{enumerate}

With this in mind, two classes of functions immediately appear as
candidates:
\begin{itemize}
\item Polynomials $\sum_{i=0}^{n} a_i\omega^i$
\item Fourier series $\sum_{k=0}^{n} a_k\sin(k\omega) +
  b_k\cos(k\omega)$
\end{itemize}

A brief analysis of both follows.

\subsection{Simplifications}
One simplification one might make is to assume that the root of the
whisker is fixed in some point on the cheek. This means that we can
let the whisker function be defined in a head-fixed coordinate system
for each whisker, with the root of the whisker at the origin. This
gives us the boundary condition
\begin{equation}
    \label{eq:bv_root}
    \text{whisker}^*(0)=\bar{0}.
\end{equation}

Manual inspection of whisker videos suggests that this is not the case
for real whiskers. However, there seems to be some point within the
cheek that stays approximately still and can be regarded as the root
of the whisker. A better model could take this into account, but that
will not be covered in this thesis.

The thickness of a whisker is not constant, but decreases as the
distance from the head increases. A simple model for this is to define
the whisker thickness as

\begin{equation}
    d(\omega) = \begin{cases}
        D-\frac{D\omega}{L},~& \omega<L\\
        0,~& \omega\ge L
    \end{cases}
\end{equation}

where $D>0$ is the thickness at the root and $L>0$ is the total length of the
whisker.

\subsection{Difference measure}
    The standard way to quantify distance between functions defined on an
    interval $\interval{a}{b}$ is to use the norm in the $\Lp[2](a, b)$
    Hilbert space. In this thesis, the norms for $p = 2, 4, 8$ will be
    used, and the impact of the choice of $p$ will be investigated. This
    means that condition 1 above says the model must be such that the
    $\Lp$ distance $\Lpnorm{\text{whisker} - \text{whisker*}}$ can be made
    arbitrarily small.

\subsection{Polynomial $\Spline{\omega}$}

    Our first and simplest candidate is the third degree polynom $\Spline{\omega}$.

    Parametrically defined as
    \begin{equation}
        (\omega,\Spline{\omega})
    \end{equation}

    This chose can be justified by the theory of beams under small deformations 
    in the area of strength of materials, after all a whisker is not to far 
    away from a beamer\cite{Hallfasthet}. The two main assumption
    for this to hold is a linear elastic material and small deformations which
    ...

    <example images compared with real whiskers>

    <<Lp analytical>>

\subsection{Sinus serie$\sum{a_n\sin (\frac{2\pi n}{L}\omega)}$}
    Another promising candidate is the sinus series.
    
    Parametrically defined as
    \begin{equation}
        (\omega,\sum{a_n\sin (\frac{2\pi n}{L}\omega)})
    \end{equation}

    
    The Fourier transform becomes $\sum{a_n\sin (\frac{2\pi n}{L}\omega)}$ when
    considering (\ref{eq:bv_root}). One can choose just a few $n$ which can be
    viewed as a few different waves propagating trough the whisker propelled by
    the whisking. 

    <<Lp analytical>>


The force that comes from the head moving on the base of the whisker is just 
sucked up by the Boundaryvalues and it will still be valid assumptions for the
elasticline to hold.

Under just a few assumptions that the material is linear elastic and the
deformations are small we have the ...

\subsection{Theoretical evalutaion}
    It is hard to analytically justify the chose of $\text{whisker}^*$ since we do not
    have an analytic model for the $\text{whisker}$.


