\subsection{Background and problem formulation}

- Video tracking (finding) is interesting
- Biologists want to track the movements of animals or parts of animals
- Most of the software available for this purpose is expensive
- Most of the software available cannot handle too many whiskers at once (clutter)
- Hedvig developed a way to track human motion in 3d using monocular video
- We want to use the ~ the same method for tracking whiskers (different mechanical model)
- What do ekeberg want and why?

KEYWORDS
- Spatial/temporal resolution

With the ever increasing power and mobility of computers, the concept of computer vision has recently seen a large increase in interest. Encompassing problems such as classification, recognition, perception and tracking, ...

In particular, biology and neuroscience researchers are interested in tracking the movements of animals or parts of animals. One such field aims to study the movements of rodent whiskers, including how they are used for perception and their correlation with neural signals. However, most whisker tracking software available today suffers a few fundamental flaws. First and foremost, they are often so expensive that not even well funded laboratories feel they can afford them. Cheaper solutions often have problems with tracking multiple whiskers at once, often requiring removal of almost all whiskers. Even if the rodent adapts to the change, it may introduce systematic errors. Some higher precision systems impose other restrictions on the experiment, such as restraining the animal or attaching motion capture markers to the whiskers, which may also cause systematic errors and not give a representative view of how the animal naturally uses the whiskers.

In general, the main difficulty in tracking and localization is to separate the object from clutter such as (...), occlusion (...), and (fill the list). (Hedvig here?) In the case of whisker tracking, the dominant problems are that whiskers often overlap and vary in size.

In 2001, Hedvig Sidenbladh developed a probabilistic method for tracking three-dimensional human motion in monocular video. This method deals with many of the problems inherent in computer vision

(Vad Hedvig introducerat som e najz) The goal of this project is to apply the same kind of method used by Sidenbladh to tracking whiskers. We will investigate whether this is feasible, and if it is we will try some different tracking models and see how they perform.

