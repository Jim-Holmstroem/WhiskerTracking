\section{Background}

%- Video tracking (finding) is interesting
%- Biologists want to track the movements of animals or parts of animals
%- Most of the software available for this purpose is expensive
%- Most of the software available cannot handle too many whiskers at once (clutter)
%- Hedvig developed a way to track human motion in 3d using monocular video
%- We want to use the same method for tracking whiskers (different mechanical model)
%- What does Ekeberg want and why?

%KEYWORDS
%- Spatial/temporal resolution

%PROSE (kind of) BELOW THIS POINT


With the ever increasing power and mobility of computers, computer vision has recently seen a large increase in interest. 
Encompassing problems such as classification, recognition, perception and tracking, application of the discipline could make many tasks easier and more efficient.

In particular, biology and neuroscience researchers are interested in tracking the movements and posture \cite{WhiskerVideography} of animals or parts of animals.
One such field aims to study the movements of rodent whiskers. However, most whisker tracking software available today suffers a few fundamental flaws.
They are either so expensive that not even well funded laboratories feel they can afford them or have problems tracking multiple whiskers at once, often requiring removal of almost all whiskers. 
Some higher precision systems impose other restrictions on the experiment, such as restraining the animal or attaching motion 
capture markers to the whiskers \cite{BadExample1}. Such restrictions may cause systematic errors and not give 
a representative view of how the animal naturally uses its whiskers.

\section{Difficulties}
In general, the main difficulty in tracking and localization is to separate the tracked object from clutter
and occlusion. In the case of whisker tracking, the dominant problems are that whiskers often overlap, 
and the relatively low spatial and temporal resolution\cite{WhiskerVideography} in highspeed video results in subpixel whiskers and motion blur.

In 2001, Hedvig Sidenbladh investigated probabilistic methods for tracking three-dimensional human motion in monocular video \cite{Hedvig}. 
Many of the problems inherent in computer vision were regarded, and the most apperent ones for the instance of whisker tracking is occlusion, 2D-3D projection and motion blur.

But also problem specific problems appear like having the whisker root constantly occluded by facial hairs gives us problem with not knowing where the whiskers is rooted\footnote{Which our algorithm assumes is given}.

\section{Approache the difficulties}
    The problems above will be addressed as they appear in the thesis.

\section{Contributions of this thesis}
The contributions in this thesis can be divided into
\begin{description}
    \item[Functional model] 
        Introducing a functional model $\text{whisker}$ and using the functional $L_p$-norm, 
        to better represent "distance" between two models, 
        for the sampling part of the particle filter.
    \item[Proof of concept]
        Providing a proof with examples on that this solution is indeed feasible.
    \item[Parameter investigation]
        Parameter analysis on the algorihtm and how the parameters effects the result\footnote{Result on the benchmark\ref{sec:benchmarks_results}.}.

\end{description}



