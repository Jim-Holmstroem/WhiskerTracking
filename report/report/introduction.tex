\section{Background}

With the ever increasing power and mobility of computers, computer
vision has recently seen a large increase in interest.  Encompassing
problems such as classification, recognition, perception and tracking,
application of the discipline could make many tasks easier and more
efficient.

In particular, biology and neuroscience researchers are interested in
tracking the movements and posture \cite{WhiskerVideography} of
animals or parts of animals.  One such field aims to study the
movements of rodent whiskers. However, most whisker tracking software
available today suffers a few fundamental flaws.  They are either so
expensive that not even well funded laboratories feel they can afford
them or have problems tracking multiple whiskers at once, often
requiring removal of almost all whiskers.  Some higher precision
systems impose other restrictions on the experiment, such as
restraining the animal or attaching motion capture markers to the
whiskers. \cite{BadExample1} Such restrictions may cause systematic
errors and not give a representative view of how the animal naturally
uses its whiskers.

\section{Difficulties}
In general, the main difficulty in tracking and localization is to
separate the tracked object from clutter and occlusion. In the case of
whisker tracking, the dominant problems are that whiskers often
overlap, and the relatively low spatial and temporal resolution in
highspeed video results in subpixel whiskers and motion
blur. \cite{WhiskerVideography}

In 2001, Hedvig Sidenbladh investigated probabilistic methods for
tracking three-dimensional human motion in monocular
video. \cite{Hedvig} Many of the problems inherent in computer vision
were regarded, and the thesis shows that powerful conclusions can be
drawn by combining multiple visual cues.\footnote{The solution in this
thesis employs only a single visual cue.}

The most apperent problems in whisker tracking is occlusion, 3D to 2D
projection inambiguities and motion blur. Other, more subtle problems
include that the whisker root is constantly occluded by facial
hairs. This gives us the problem of not knowing where the whiskers are
rooted, making the whiskers more difficult to model.

\section{Approache the difficulties}
The problems above will be addressed as they appear in the thesis.

\section{Contributions of this thesis}
The contributions in this thesis can be divided into
\begin{description}
\item[Functional model] Introducing a functional model
  $\text{whisker}$ and using the functional $L_p$-norm, to better
  represent "distance" between two models, for the sampling part of
  the particle filter.
\item[Proof of concept] Providing a proof with examples on that this
  solution is indeed feasible.
\item[Parameter investigation] Parameter analysis on the algorihtm and
  how the parameters effects the result\footnote{Result on the
    benchmark\ref{sec:benchmarks_results}.}.

\end{description}
