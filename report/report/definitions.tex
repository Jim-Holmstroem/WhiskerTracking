=============== Invariances ==============
Section about invariances and why they are needed in localization.

The "filter" must give the same response invariant of the position in the image 
and its often wanted to have it rotational and scale invariant (if your not 
intending to measure things like angles or size that is)

============== Data ========================
= Image
In a computer an image is represented as an array of 
numbers, mostly 8bit numbers ranging between [0,255] 

[should one use N or [0,255] instead of Z?]
== RGB image is a function;
image: ZxZ=pos->ZxZxZ=color (mapping each position into an RGBcolor)
that is foreach pixel a value for Red,Green,Blue is defined.
But one might as well see it as a list of pixels Z^5=<pos,color>

== Grayscale image is in the same way as an RGB image but without the RGB part
just a function;
image: ZxZ->Z
or alternatively list of Z^3=<pos,value>
== Video
Ordered list of images


== Feature

== Featurespace
One configuration of the object corresponds to one point in the featurespace
<example with a plot>


== Hypothesis
Explain: Hypotesis (in the context of PF), sample, DOF?, 

== Prop. function as a collection of points with attached weight.


== Dynamic system
    It's a tuple <space,update,time>
    Any mechanical system following newton physics (or relativistic for that matter) can be considered to be an dynamic system
    in our case the tuple whould be <feature_space,update_rule (the one we are trying to preprocess data to approximate),time>



