%TODO 
%\section{Invariances}
%Section about invariances and why they are needed in localization.

%The "filter" must give the same response invariant of the position in the image 
%and its often wanted to have it rotational and scale invariant (if your not 
%intending to measure things like angles or size that is)

\section{Data}
\subsection{Image}
In a computer an \emph{image} is represented as an array of numbers, mostly 8bit integers ranging between $\left[0,255\right]$ 

\subsubsection{RGB image}
A color image can be defined as
\begin{equation}
    \begin{array}{ccc}
    RGBimage : \NN^2 &\rightarrow& \NN^3\\
    RGBimage : position&\rightarrow&color
    \end{array}
\end{equation}
Which basically is foreach position in the image we have a response color.
An alternitve represenation whould be $\NN^5=\langle pos,value\rangle$.

The RGB image space is denoted by
\begin{equation}
    RGBimage\in \IS_{RGB}
\end{equation}

\subsubsection{Grayscale image}
A grayscale image can, similary to RGB images, be defined as a function
\begin{equation}
    \begin{array}{ccc}
    I : \NN^2 &\rightarrow& \NN\\
    I : position&\rightarrow&intensity
    \end{array}
\end{equation}
or alternatively $\NN^3=\langle pos,intensity\rangle$.

Finally the $image$-space is denoted by
\begin{equation}
    I \in \IS
\end{equation}

\begin{example}
    test
    %\includegraphics{smiley/smiley.gif} 
\end{example}

\subsubsection{Video}
Ordered list of images

\subsubsection{Hypothesis}
Explain: Hypotesis (in the context of PF), sample, DOF?, 

\subsubsection{Prop. function as a collection of points with attached weight.}


\subsubsection{Dynamic system}
    It's a tuple <space,update,time>
    Any mechanical system following newton physics (or relativistic for that matter) can be considered to be an dynamic system
    in our case the tuple whould be <feature\_space,update\_rule (the one we are trying to preprocess data to approximate),time>



