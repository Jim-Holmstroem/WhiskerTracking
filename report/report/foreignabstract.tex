
Intresset för att studera morrhår hos gnagare har på senare tid ökat, speciellt inom neurofysiologin.
Som ett resultat av detta finns det ett behov av automatisk följning av morrhår.
Nuvarande kommersiella lösningar är antingen extremt dyra, sätter begräsningar på experimentuppställningen
eller misslyckas i stökiga miljöer eller när överlappning förekommer.

Denna avhandling bidrar med en enkel implementation av ett probabilistiskt följningssystem. 
Lösningen använder sig av 
en teknik som kallas \emph{partikelfilter} för att propagera en morrhårmodell mellan bildrutor 
i höghastighetsvideo.
För varje bildruta förutspås nästa tillstånd genom att fråga en förtränad databas 
och filtera svaret genom partikelfiltret. 
Implementationen är skriven i Python 2.6 och använder sig av programbiblioteken NumpPy och SQLite3.

Testresultaten indikerar att metoden är rimlig. 
Med endast en grovt snarlik databas lyckades trackern 
tracka multipla morrhår, dock endast under korta sekvenser i taget. 
Bättre träningsdata, såsom handmarkerad riktig data, skulle kunna förbättra resultatet väsentligt.


\textbf{Nyckelord:} Följning, Multipla, Morrhår, Partikelfilter, Övergångsdatabas, Modelevaluering, Proof-of-Concept
