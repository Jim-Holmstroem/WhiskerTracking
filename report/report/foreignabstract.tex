
Intresset för att studera morrhår på gnagare har på senare tid ökat, speciellt inom neurofysiologin.
Som ett result av detta finns det ett behov av automatisk följning av morrhår.
Nuvarande kommersiella lösningar är antingen extremt dyr, sätter begräsningar på experiment uppställningen
eller misslyckas i stökiga miljöer eller under överlappning.

Arbetets mål är en proof-of-concept implementering av ett statiskt följningssystem. 
Lösningen använder sig av 
en teknik känd som partikelfilter för att propagera en morrhårmodell mellan bildrutor 
på höghastighetsvideo.
För varje bildruta förutspås nästa tillstånd genom att fråga en förtränad databas 
via partikelfiltret. 
Implementationen är skriven i Python 2.6 och använder sig av programbiblioteken NumpPy\cite{NumPy} and SQLite3\cite{SQLite3}.

Testresultaten indikerar att metoden är rimligt. 
Med endast en grovt snarlik databas lyckades trackern 
tracka multipla morrhår, fast bara under korta sekvenser i taget. 
Bättre träningsdata, som hand-markerad riktig data, skulle kunna ändra resultatet enormt.


\textbf{Nyckelord:} Följning, Multipla, Morrhår, Partikelfilter, Övergångsdatabas, Modelevaluering, Proof-of-Concept
