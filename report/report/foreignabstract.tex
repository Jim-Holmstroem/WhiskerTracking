
Intresset för att studera morrhår på gnagare har på senare tid ökat, speciellt inom neurofysiologi.
Som ett result finns det ett behov av automatisk följning av morrhår.
Nuvarande kommersiella lösningar är antingen extremt dyr, sätter begräsningar på experiment uppställningen
eller misslyckas i stökiga miljöer eller under överlappning.
Arbetets mål är en proof-of-concept implementering av ett statiskt följningssystem. Lösningen använder sig av 
en teknik känd som partikelfilter för att propagera en morrhårmodell mellan bildrutor på höghastighetsvideo.
För varje bildruta förutspås nästa tillstånd genom att fråga en förtränad databas via partikelfiltret.

\textbf{Nyckelord:} Följning, Multipla, Morrhår, Partikelfilter, Övergångsdatabas, Modelevaluering, Proof-of-Concept
