\section{Introduction} The core of the tracking engine is a technique
known as the \emph{particle filter}.  The particle filter is a kind of
Bayesian filtering where one uses discrete hypotheses, also known as
\emph{particles}, to approximate continuous PDF \cite{ProbRob}.  It
builds upon the theory of \emph{Markov processes} and the \emph{hidden
Markov model}.


\begin{figure} \centering
  \includegraphics[width=0.175\textwidth]{mp-graph.pdf}
  \caption{Schematic image of a Markov process.}
  \label{fig:hmm-graph}
\end{figure}


\section{Markov processes} A Markov process is a special case of a
stochastic process. For a Markov process, the next state depends only
on the present state and not on past states.  For this reason, a
Markov process is often said to be ``forgetful''.

In mathematical terms, a Markov process satisfies the following:
\begin{equation} p\left(Z_t|Z_{t-1} \wedge Z_{t-2} \wedge \dots \wedge
Z_0\right) = p\left(Z_t|Z_{t-1}\right),
\end{equation} $p\left(Z_t|Z_{t-1} \wedge Z_{t-2} \wedge \dots \wedge
Z_0\right)$ is the probability that the system will have state $Z_t$
at time $t$, given that the previous states where $Z_{t-1},
Z_{t-2},\dots, Z_0$.

\begin{figure} \centering
  \includegraphics[width=0.35\textwidth]{hmm-graph.pdf}
  \caption{Schematic image of a hidden Markov model.}
  \label{fig:hmm-graph}
\end{figure}

\section{The Hidden Markov Model}

The working principle of the particle filter is based on the
\emph{hidden Markov model} (HMM).  A HMM describes a Markov process
where we cannot measure the state directly - it is
``hidden''\cite{EncyclopediaMachineLearning}.  Instead we obtain an
\emph{observation} $I$\footnote{In this thesis, the observation is
always a grayscale \emph{image}, therefore the observation is denoted
$I$.}  of the state. This \emph{perception} is generally
non-deterministic, so we need to denote it as $p(I_t|Z_t)$ which is
the probability that we will observe $I_t$ if the state is $Z_t$.

\section{The Curse of Dimensionality} A phenomenon that becomes
apparent in high-dimensional spaces is the so-called ``Curse of
dimensionality'' \cite{EncyclopediaMachineLearning}.  The problem is
that the search volume grows exponentially with the number of
dimensions.  It originates from the fact that we need $\Ordo{C^n}$
samples to obtain a sample density of $C$ in a $n$-dimensional space.

The first consequence of this is that in order to approximate a
high-dimensional function one needs orders of magnitude more samples.

The other drawback with high dimensional space is the large
``borders'' of the sample-set compared to lower dimensional space
which results in orders of magnitude higher chance for an point one
want to approximate to fall outside the sample-set and needs to be
extrapolated instead of the better alternative of interpolation.

% For example, consider an interval on a number line, where we cover
% the middle third of the interval with samples.  If we randomly
% select a number from the interval, there is a one in three chance
% that our selected number is in the region with samples.  Take this
% to two dimensions, and we have covered only a ninth of the space,
% and so on. This means that the chance that the point we want to
% approximate falls outside the sample set is orders of magnitude
% higher in higher dimensional spaces, compared to lower dimensional
% ones.  We need more samples in order to be able to interpolate
% between samples.

\begin{example} Figure \ref{fig:curse-of-dimensionality} shows 128
randomly scattered points in 1, 2 and 3 dimensions. Notice how the
density decreases with increasing dimension.
  \begin{figure}
    \begin{tabular}{rcl}
      \includegraphics[scale=0.3,trim=4cm 4cm 4cm 4cm]{1D.pdf}&
      \includegraphics[scale=0.3,trim=4cm 4cm 4cm 4cm]{2D.pdf}&
      \includegraphics[scale=0.3,trim=4cm 4cm 4cm 4cm]{3D.pdf}
    \end{tabular}
    \caption{Plots of $128$ scattered samples in $1$, $2$ and $3$
dimensions, respectively.}
    \label{fig:curse-of-dimensionality}
  \end{figure}
\end{example}

\begin{example} For a $16$ DOF model one needs $10^{16}=10$
quadrillion datapoints to acquire a density of $10$ samples per unit
volume. Millions of gigabytes would be needed just to store the
samples.
\end{example}

\begin{example} In 2 dimensions it is sometimes feasible to use an
exhaustive search.  An example of this is the Hough transform
\cite{DigitalImageProcessing}, where the search is done through the
$\rho\theta$ space of line responses on images.
\end{example}

\subsection{Overcoming the Curse} One way to overcome the curse in the
context of tracking is to perform a directed search. Let the search be
in an $n$ dimensional space with a grid of $g$ grid lines in each
direction.

\begin{enumerate}
\item Use the information about the most recent\footnote{In the
    Bayesian case, the most recent estimate} location and assume that
  the tracked object cannot travel more than $R < g$ grid steps in one
  time step.  This reduces the volume of the (discrete) search space
  from $\Ordo{g^n}$ to $\Ordo{R^n}$.
      % \begin{proof}
      %   Change the coordinate system to generalized spherical
      %   coordinates $(r,\phi_1,\phi_2,...,\phi_{N-1})$ and fixate
      %   $r$ foreach $r<R$ which gives us a $(N-1)$ dimensional
      %   search surface, sum up the fixated $r$'s to get the total
      %   search space.
      % \end{proof}
\item With prior knowledge of how the tracked objects move
  \footnote{Such as the state transition probabilites $\cprobnext{Z}$
    in a HMM} we can direct our search to specific regions in the
  state space, depending on how probable it is for the tracked object
  to be located there. This reduces the size of the search space
  depending on how sure we are of the previous state.
\end{enumerate}

% This phenomen of having hue searchspace in highdimensinonal space is
% fairly common and have the name "Curse of dimensionality".

% One way to somewhat overcome this emptyness in space is to have a
% dynamic (active?) algorithm that adopts the sample density according
% to the models relative frequency in that area and this results in,
% for a given amount of samples, its more likely for an sampled model
% to occure in a more densly pre-sampled (pre-sampled?) area, in fact
% this method when doing it right (ideally) gives: given a set of
% samples the overall density for all the samples in the
% sampledatabase would be optimal) [proof for this will be given in
% <blabla>] ... (use the world hypotesis instead of sample, or perhaps
% sampled-hypotesis) (below is perhaps somewhat redudant, but one can
% pick bits and pieces from both) So having a bias towards trying out
% more plausible hypotesis for the models innerstate is better than
% doing a naive exhausted search tru the entire feature space, this
% could only be used if you have $\le$ 3 DOF as in for example finding
% straight edges with houghtransformation[ref].  ...  One method that
% uses prior knowledge of the models PDF and a pretrained database
% containing knowledge on how to approximate the models PDF in the
% next timestep is particle filter which is an (instance?) of the
% ideal bayesian filter.


% """ The "Curse of dimensionality", is a term coined by Bellman to
% describe the problem caused by the exponential increase in volume
% associated with adding extra dimensions to a (mathematical)
% space. One implication of the curse of dimensionality is that some
% methods for numerical solution of the Bellman equation require
% vastly more computer time when there are more state variables in the
% value function.

% For example, 100 evenly-spaced sample points suffice to sample a
% unit interval with no more than 0.01 distance between points; an
% equivalent sampling of a 10-dimensional unit hypercube with a
% lattice with a spacing of 0.01 between adjacent points would require
% 1020 sample points: thus, in some sense, the 10-dimensional
% hypercube can be said to be a factor of 1018 "larger" than the unit
% interval. (Adapted from an example by R. E. Bellman, see below.)
% """

% from: "R. Bellman, Adaptive control Processes, p.94, Princeton
% University Press, NJ, 1961."

% “In view of all that we have said in the foregoing sections, the
% many obstacles we appear to have surmounted. What casts the pall
% over our victory celebration? It is the curse of dimensionality, a
% malediction that has plagued the scientist from earliest days.”

% """
%
% Number of states grows exponentially in n (assuming fixed number of
% discretization levels per coordinate)
%
% """

%
% """ One solution on how to make the effects of the curse of
% dimensionality is to make a directed search..  """
%

%
% """ Bellmans dynamic programming (DP) requires knowledge of
% transition probablities of the dynamic system from ones state to the
% next """
%

\section{The Particle Filter}

One naive way to compute the state $Z_t$ would be to perform an
exhaustive search in the state space, and select the state for which
$\cprob{I_t}{Z_t}$ is maximised.<ref:COD>

The particle filter is a technique for reducing the search
space<ref:COD>. It uses a finite set $X_t$ of hypotheses to
approximate the PDF $\cprobnext{Z}$ of a HMM. The hypotheses $X_t$ are
also refered to as \emph{particles}, thereby the term ``particle
filter''.

\begin{figure}
  \centering
  \includegraphics[width=0.8\textwidth]{hmm-pf-graph.pdf}
  \caption{Schematic image of the particle filter alongside a HMM.}
  \label{fig:hmm-graph}
\end{figure}

Figure \ref{fig:hmm-graph} shows the principle of the particle filter
working alongside a hidden Markov model. The following is the core
function of the particle filter:

\begin{quote}
  \emph{The particle filter attempts to approximate the PDF
    $\cprobnext{Z}$ as a set $X_t$ of discrete hypotheses.}
\end{quote}

More particles mean greater accuracy, since the PDF can then be
approximated more closely. However, using many particles increses
computational cost.  Therefore the number of particles is an important
trade-off.  The particle filter employs a few tricks to \emph{filter}
the hypotheses, tending to keep probable ones and throwing improbable
ones away, in order to intelligently reduce the number of particles
needed for a good approximation. The filter works in four steps:

\begin{description}
\item[Prediction] The hypotheses $X_{t-1}$ are updated in the
  \emph{prediction} step to an approximation $\bar{X}_t$ of
  $\cprobnext{Z}$.  This is done by drawing new samples
  $\cprobnext{x}$, for each $x_{t-1}$ in $X_{t-1}$.
\item[Perception] By measuring the state of the system, we gain an
  \emph{observation} $I_t \sim \cprob{I_t}{Z_t}$ of the state $Z_t$.
\item[Filtering] The observation $I_t$ of the system is then used for
  filtering bad hypotheses out of $\bar{X}_t$.  We draw samples $X_t$
  from $\bar{X}_t$ with probabilities given by
  $\cprob{I_t}{\bar{X}_t}$. The result will be a surjection, where
  $X_t$ will be a subset of $\bar{X}_t$ where more probable hypotheses
  appear multiple times.<ref:problem without sigma> For this reason,
  this is also known as the \emph{resampling} step. The set $X_t$ is
  the \emph{belief}, our approximation of $\cprobnext{Z}$.
\item[Selection] Finally, we produce a single hypothesis $x_t$ from
  $X_t$ as our \emph{estimate} of the state $Z_t$. Assuming $X_t$ is a
  good approximation of $\cprobnext{Z}$, and that $\cprobnext{Z}$ is
  unimodal, the means value of $X_t$ is a good estimate since it
  approximates the expectation of $\cprobnext{Z}$.  If $\cprobnext{Z}$
  is multimodal, however, the mean could be a bad estimate since the
  expectation may be very improbable.
\end{description}

The reason why this works is the following theorem, which is Theorem
3.1 in \cite{Hedvig} with some modifications to notation:

\begin{theorem}
  Given the PDFs $\cprobnext{x}$, $\cprob{I_t}{x_t}$ and
  $\cprob{x_{t-1}}{I_{t-1} \wedge \dots \wedge I_0}$, the PDF
  $\cprob{x_t}{I_t \wedge \dots \wedge I_0}$ can be expressed as

\begin{equation}
  \cprob{x_t}{I_t \wedge \dots \wedge I_0} = \kappa \cprob{I_t}{x_t} \int{ \cprobnext{x}\cprob{x_{t-1}}{I_{t-1} \wedge \dots \wedge I_0} \mathrm{d}x_{t-1}},
  \label{thm:pf-grand}
\end{equation}
where $\kappa$ is a normalization constant

For the proof of this, see \cite{Hedvig}.

\end{theorem}

Let's sketch out the connection between this and what we've done so
far. For a more rigorous derivation, see chapter 4 of \cite{ProbRob}.

In this thesis, $\cprob{x_t}{I_t \wedge \dots \wedge I_0}$ is
approximated as the set $X_t$ of hypotheses. Substituting this into
\eqref{thm:pf-grand} and replacing the integral with a sum yields

\begin{equation}
  X_t = \kappa \cprob{I_t}{x_t} \sum { \cprobnext{x} X_{t-1} }.
\end{equation}

Note that this is horrible abuse of notation, but from here we can
take the step to

\begin{equation}
  X_t \sim \cprob{I_t}{x_t} \cprobnext{x}
\end{equation}

\subsection{The Particle Filter algorithm}
\begin{table}
  \begin{codebox}
    \Procname{$\proc{Particle-Filter} (X_{t-1},I_t)$}
    \li $\bar{X}_t \gets \emptyset$
    \li \ForEach $x_{t-1} \in X_{t-1}$
    \li \Do
    \li $x_t \gets \proc{Predict}(x_{t-1})$
    \li $w \gets \proc{Importance}(x_t,I_t)$
    \li Append $\left<x_t, w\right>$ to $\bar{X}_t$
    \End
    \li
    
    \li $X_t \gets \emptyset$
    
    \li \While $\abs{X_t} < \abs{\bar{X}_t}$
    \li \Do
    \li Take $\left<x_t, w\right>$ from $\bar{X}_t$ with probability $\propto w$
    \li Append $x_t$ to $X_t$
    \End
    \li \Return $X_t$
  \end{codebox}
  \caption{The particle filter algorithm.}
  \label{alg:pf}
\end{table}

Table \ref{alg:pf} shows the particle filter algorithm. Note that the
functions \textsc{Predict} and \textsc{Importance} are unspecified -
they are problem specific. They correspond to the PDF $\cprobnext{x}$
and $\cprob{I_t}{x_t}$, respectively.

% We draw a set $\bar{X}_t = \xtmN{x}{t}{i}{N}$ of samples from
% $p$. These samples roughly represent a PDF for the current state
% $x_t$, but we have yet to consider our observation.  Therefore, we
% will create a new PDF weighted by how probable the observation $z_t$
% is. For each $x_t^m$, we let $w_t^m := q\left(z_t |
%   x_t\right)$. This defines a discrete PDF where $x_t^m$ is assumed
% with a probability proportional to $w_t^m$. From this final
% distribution we again draw $N$ samples $X_t$, which will be our
% estimate of the current state.  The elements $x_t^m$ are referred to
% as \emph{particles} and the set $X_t$ as the \emph{belief at time
%   $t$}.

In this thesis, the real $x_{t-1}$ is not known. Rather $x_{t-1}$ is
estimated with a set $X_{t-1}$ of $N$ particles.

\section{Visual Cues}
\begin{quote}
    The biggest problem with computer vision is that computers do not have
    vision, only a data input device in the form of a camera.
\end{quote}
A \emph{visual cue} is an image transformation $\phi$ that extracts
some property of the image, such as intensity
\footnote{In our case this is possible since we have an homogenous object
in the form of a backlit rodent}, edges, ridges\cite{Hedvig} or different 
refinements as in section \ref{prep-real}.


\begin{theorem} %TODO svagare?
  \label{thm:response_max}
  Let $f$ be a positive Riemann function with compact support,
  defined on a set $\Omega$. Then
  \begin{equation}
    \argmax{\bar{e}}
    \left( \int\limits_{\Omega}{f(\bar{x})f(\bar{x}-\bar{e})} \right)
    =0
    \label{thm:response-max-eqn}
  \end{equation}
\end{theorem}
\begin{proof}
  Firstly define the window function as

  \begin{IEEEeqnarray*}{s"r?c?l}
    & W_a^b(x) & = &
    \begin{cases}
      0,~& x<a\\
      1,~& a \leq x \leq b\\
      0,~& x>b
    \end{cases}\\

    Multiplication: &
    (W_a^bW_c^d)(x) & = & W_{\max(a,c)}^{\min(b,d)}(x)\\

    Translation: &
    W_a^b(x-e) & = & W_{a+e}^{b+e}(x)\\

    Integration: &
    \int\limits_{\RR}{W_a^b(x)dx} & = & \Theta(b-a).
  \end{IEEEeqnarray*}
  
  \begin{IEEEeqnarray*}{L?c?l}
    \argmax{e}\left( \int{W_a^b(x)W_a^b(x-e)} \right) &=& \\
    \argmax{e}\left( \int{W_a^b(x)W_{a+e}^{b+e}(x)} \right) &=&\\
    \argmax{e}\left( \int{W_{\max(a,a+e)}^{\min(b,b+e)}(x)} \right) &=&\\
    \argmax{e}\left( \Theta(\min(b,b+e)-\max(a,a+e)) \right) & = & 0.
  \end{IEEEeqnarray*}
  
  This trivially holds for superposition of windows, since all windows
  will scale and translate the same way.  With a finite support, $e=0$
  is the only solution. Additionaly, this also holds in higher finite
  dimensions since we can just repeat the process one dimension at a
  time.
    
  All riemann functions can be written as a superposition of windows
  like this
  \begin{equation*}
    f(x)=\sum{c_iW_{a_i}^{b_i}(x)},
  \end{equation*}
  
  and therefore \eqref{thm:response-max-eqn} holds for any Riemann function $f$ with finite support.
  
\end{proof}

\section{Model}

%"
%==Model selection
%Model selection is the process of choosing an appropriate mathematical model
%from a class of models.
%" - Encyclopedia 

A whisker can be modeled as a function:

% In the following chapter we will formalize the whisker and discuss
% our model selection\cite{EncylopediaMachineLearning} process.

% First of we formalize the whisker by the definition

\begin{definition}
    Let 
    \begin{equation}
    \begin{split}
        \text{whisker} : \RR^+ &\rightarrow \RR^2 \\
                  \omega&\mapsto \text{whisker}(\omega)
    \end{split}
    \end{equation}
\end{definition}
where $\omega$ is a coordinate along the length of the whisker and
$\text{whisker}(\omega)$ is a point in the $x$-$y$ plane.

The tracking problem is then to find a function $\text{whisker}^*$
that approximates $\text{whisker}$. The function class used in a
model must satisfy the following conditions:

\begin{enumerate}
\item It must be able to approximate the whisker sufficiently
  well.

  \begin{example}
    A straight line will not suffice since the $\text{whisker}$ is
    generally curved and straight lines can not fill that partition of
    the space.
  \end{example}
  
\item The functions must be $C_1$ and be possible to represent with a
  finite number of parameters.
\end{enumerate}

With this in mind, two classes of functions immediately appear as
candidates:
\begin{itemize}
\item Polynomials $\sum_{i=0}^{n} a_i\omega^i$
\item Fourier series $\sum_{k=0}^{n} a_k\sin(\frac{k\pi\omega}{L}) +
  b_k\cos(\frac{k\pi\omega}{L})$
\end{itemize}

A brief analysis of both will follow, after a discussion of
simplifications and difference measures..

\subsection{Simplifications}
One simplification one might make is to assume that the root of the
whisker is fixed in some point on the snout. This means that we can
let the whisker function be defined in a head-fixed coordinate system
for each whisker, with the root of the whisker at the origin. This
gives us the boundary condition
\begin{equation}
    \label{eq:bv_root}
    \text{whisker}^*(0)=\bar{0}.
\end{equation}

Manual inspection of whisker videos suggests that this is not the case
for real whiskers. However, there seems to be some point within the
snouts that stays approximately still and can be regarded as the root
of the whisker. A better model could take this into account, but that
will not be covered in this thesis.

The thickness of a whisker is not constant, but decreases as the
distance from the head increases. A simple model for this is to define
the whisker thickness as

\begin{equation}
    d(\omega) = \begin{cases}
        D-\frac{D\omega}{L},~& \omega<L\\
        0,~& \omega\ge L
    \end{cases}
\end{equation}

where $D>0$ is the thickness at the root and $L>0$ is the total length of the
whisker.

\subsection{Difference measure}
The standard way to quantify distance between functions defined on an
interval $\interval{a}{b}$ is to use the norm in the
$\Lp[2](\interval{a}{b}, 1)$ Hilbert space. In this thesis, the norms
for $p = 2, 4, 8$ will be used, and the impact of the choice of $p$
will be investigated. This means that condition 1 above says the model
must be such that the $\Lp$ distance $\Lpnorm{\text{whisker} -
  \text{whisker*}}$ can be made sufficiently small.

\subsection{Polynomial $\Spline{\omega}$}
The first and simplest candidate is the polynomials. Manual tests in
MATLAB indicate that three terms are enough to approximate a whisker
well enough that the difference is not visible to the naked eye. A
whisker function can therefore be modeled as a third degree polynomial
$\Spline{\omega}$, also known as a \emph{spline}.

These define parameterized curves in the $xy$ plane as
\begin{equation}
  (\omega,\Spline{\omega})
\end{equation}

This choice can be justified by comparing with the theory of beams
under small deformations in solid mechanics. After all, a whisker is
not too different from a beam. The two main assumptions for this to
hold is that deformations are small and that the effects of motion on
whisker deformation are negligible. \cite{Hallfasthet} This may not
quite be the case, but a spline is still capable of approximating the
momentaneous shape of a whisker well enough.

\subsection{Sine series $\sum a_k\sin \left(\frac{k\pi\omega}{L}\right)$}
Another promising candidate is the Fourier series. Considering
\eqref{eq:bv_root}, the Fourier series is reduced to a sine series
$\sum a_k\sin \left(\frac{k\pi\omega}{L}\right)$. However, such a
series will always be zero at $\omega = L$, and is therefore not a
reasonable model for whiskers. To address this, one could instead use
quarter-periods: $\sum a_k \sin\left( \frac{k\pi\omega}{2L}\right)$,
but at the cost of losing the orthogonality property of the sine
basis. Manual tests in MATLAB indicate that such a series to third
order performs approximately the same as a spline.

These define parameterized curves in the $xy$ plane as
\begin{equation}
    (\omega,\sum{a_n\sin (\frac{2\pi n}{L}\omega)})
\end{equation}


The choice of a sine series can be justified by comparing with the
theory of stiff strings in analytical mechanics. The movement equation
for a stiff string is a partial differential equation containing the
second and fourth spatial derivatives.\footnote{Source: Solving
  exercise 7.10 of \cite{VarKalk} using variational calculus.} A
classic separation of variables solution to the equation would be a
sine series for the spatial part.

\subsection{Theoretical evaluation}

It is hard to theoretically justify the choice of $\text{whisker}^*$
model since we do not have an analytic model for the $\text{whisker}$.

\begin{figure}
  \centering
  \includegraphics[width=0.45\textwidth]{rat-splines.png}
  \includegraphics[width=0.45\textwidth]{rat-sines.png}
  \caption{Comparison of whisker models. Left: Splines, Right: Sine series}
  \label{fig:model-comparison}
\end{figure}

For this thesis, the spline model was used in the tracking engine. The
main reason for this is that it is slightly simpler than the sine
model. It also is rather easy to get an intuitive grasp of how the
parameters affect the curve shape - the $\omega^3$ term mostly affects
the tip of the whisker while the $\omega$ term affects the overall
orientation.

