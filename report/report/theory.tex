\newcommand{\xtmN}[4]{
  \left\{#1_{#2}^{#3}\right\}_{#2=1}^{#4}
}
\newcommand{\interval}[2]{
  \left[#1, #2\right]
}
\newcommand{\bel}[1]{
  bel\left(#1\right)
}

\subsection{The particle filter}
\subsubsection{Introduction}
The core of tracking engine used is a technique known as the \emph{particle filter}. The particle filter is a kind of Bayesian filtering where one uses discrete hypotheses to approximate continuous probability distributions \cite{ProbRob}. The main idea can be outlined as follows.

Suppose we have a system described by the state $x_t$ at time $t$. $x_t$ can be thought of as a vector of state parameters. Suppose that we know the previous state $x_{t-1}$ and have an observation $z_t$ of the system at the current time. In general, it is difficult to accurately determine $x_t$ from the observation alone, and the observation may suffer from interference. Therefore, we cannot directly read $x_t$ from $z_t$. However, we can estimate $x_t$ if we know the following things about the system:

\begin{itemize}
\item A probability function $p$, where $p\left(x_t | x_{t-1}\right)$ is the probability that the current state is $x_t$ if the previous state was $x_{t-1}$.
\item A probability function $q$, where $q\left(z_t | x_t\right)$ is the probability that we observe $z_t$ if the current state is $x_t$.
\end{itemize}

Using $p$ and knowing $x_{t-1}$, we can generate a set of hypotheses $\xtmN{\bar{x}}{t}{m}{N}$ for the current state $x_t$. Using $q$ and knowing $z_t$, we can evaluate how likely the hypotheses are. If it is likely to observe $z_t$ if the current state is $x_t$, then $x_t$ is probably a good estimate of the current state. With this information, we select the most probable hypotheses and let them be our estimate of the current system state.

The particle filter works recursively in two steps:

\begin{enumerate}
\item The \emph{sampling step} is the generation of the hypotheses, known as \emph{particles}, from the previous state $x_{t-1}$. The belief $\bel{x_{t-1}}$, see below, is used as an estimate for $x_{t-1}$. What makes this recursive is the fact that $\bel{x_t}$ is calculated using $\bel{x_{t-1}}$.

\item The \emph{resampling step} is the final selection of the most probable particles. After resampling the set $\bar{X_t} := \xtmN{\bar{x}}{t}{m}{N}$ we get the \emph{belief} $\bel{x_t}$, which often includes multiple copies of the most probable particles. This set is used as an estimate for $x_t$, and is used to estimate $x_{t+1}$.
\end{enumerate}

In the next section, the particle filter algorithm will be stated. Implementing the filter in itself is only a matter of implementing the stated pseudocode, and is not difficult. The difficult part is designing the probability functions $p$ and $q$ for the given system. A probabilistic implementation of these functions is proposed in chapters TODO.


\subsubsection{Formal description of the particle filter}
Here the particle filter algorithm is stated. An elaboration on what this actually does and why is offered below.
%TODO pseudo algorithm for the 
%Sample instead of draw

\begin{codebox}
\Procname{$\proc{Distribution-Sample}(X_t,p)$}
\li \ForEach $\id{x_t}$ \In $\id{X_t}$
\li     \Do
            $sample \id{x_{t+1}} \sim p\left(x_{t+1}|x_t\right)$
        \End
\li \Return $\id{X_{t+1}}$
\end{codebox}
\begin{codebox}
\Procname{$\proc{Importance}(X,q,z)$}
\li \ForEach $\id{x}$ \In $\id{X}$ 
\li     \Do
            $w \gets q\left(z|x\right)$
        \End
\li \Return $W$
\end{codebox}
\begin{codebox}
\Procname{$\proc{Weighted-Sample}(X,W)$}
\li \ForEach $x$ \In $X$
\li     \Do
            $sample ~ x' \propto W$   
        \End
\li \Return $X'$
\end{codebox}
\begin{codebox}
\Procname{$\proc{Particle-Filter} (X_{t-1},z_t)$}
\li $X_t \gets \proc{Distribution-Sample}(X_{t-1},p)$
\li $W_t \gets \proc{Importance}(X_{t-1},q,z_t)$
\li $X_t \gets \proc{Weighted-Sample}(X_t,W_t)$
\li \Return $X_t$
\end{codebox}

%\begin{codebox}
%\Procname{$\proc{Particle-Filter} (X_{t-1}, z_t)$}
%\li Let $\bar{X_t} = \emptyset$.
%\li For $m:=1$ to $\left|X_{t-1}\right|$:
%\li \Do Draw $x_t^m$ with probability $p\left(x_t^m | x_{t-1}^m\right)$.
%\li Let $w_t^m := q\left(z_t | x_t^m\right)$.
%\li Add $(x_t^m, w_t^m)$ to $\bar{X_t}$.
%\End
%\li Let $X_t := \emptyset$.
%\li For $m:=1$ to $\left|X_{t-1}\right|$:
%\li \Do
%    Draw $x_t^m$ with probability proportional to $w_t^m$
%    \li Add $x_t^m$ to $X_t$.
%    \End
%\li Return $X_t$.
%\end{codebox}


We draw a set $\bar{X}_t = \xtmN{x}{m}{t}{N}$ of samples from $p$. These samples roughly represent a probability distribution for the current state $x_t$, but we have yet to consider our observation. Therefore, we will create a new probability distribution weighted by how probable the observation $z_t$ is. For each $x_t^m$, we let $w_t^m := q\left(z_t | x_t\right)$. This defines a discrete probability distribution where $x_t^m$ is assumed with a probability proportional to $w_t^m$. From this final distribution we again draw $N$ samples $X_t$, which will be our estimate of the current state. The elements $x_t^m$ are referred to as \emph{particles} and the set $X_t$ as the \emph{belief at time $t$}.

In this thesis, $x_{t-1}$ is not known. Rather $x_{t-1}$ is estimated with a set $X_{t-1}$ of $N$ particles. In this case, $x_t^m$ is sampled with probability $p\left(x_t^m | x_{t-1}^m\right)$.



\subsubsection{Curse of dimensonality}


% """
%The "Curse of dimensionality", is a term coined by Bellman to describe the
%problem caused by the exponential increase in volume associated with adding
%extra dimensions to a (mathematical) space. One implication of the curse of
%dimensionality is that some methods for numerical solution of the Bellman
%equation require vastly more computer time when there are more state variables
%in the value function.

%For example, 100 evenly-spaced sample points suffice to sample a unit interval
%with no more than 0.01 distance between points; an equivalent sampling of a
%10-dimensional unit hypercube with a lattice with a spacing of 0.01 between
%adjacent points would require 1020 sample points: thus, in some sense, the
%10-dimensional hypercube can be said to be a factor of 1018 "larger" than the
%unit interval. (Adapted from an example by R. E. Bellman, see below.)
% """

from:
"R. Bellman, Adaptive control Processes, p.94, Princeton University Press, NJ,
1961."

“In view of all that we have said in the foregoing
sections, the many obstacles we appear to have
surmounted. What casts the pall over our victory
celebration? It is the curse of dimensionality, a
malediction that has plagued the scientist from
earliest days.”


show curse of dimensionallity with a simple like in bik12.lecture7
1D->2D
either maintain density samples increases drastically
or maintain samples and density decreases drastically
2D->3D
even worse

To have a sample density of C we need $\Ordo{C^n}$ samples for n-dimensional data,
that is the number of samples grows exponential on the number of dimensions.
And a consequence of this is that in order to approximate some
function defined in higher dimensional space one needs many more samples.

One more thing with the high dimensional space is the large boundaries 
of the sample-set then from lower dimensional space which results in orders of
magnitude higher chance for an point one want to approximate to fall outside
the sample-set and needs to be extrapolated instead of interpolate which is
much better.

% """
%
% Number of states grows exponentially in n (assuming fixed number of
% discretization levels per coordinate)
%
% """

%
%"""
% One solution on how to make the effects of the curse of dimensionality is to
% make a directed search..
%"""
%

%
%"""
% Bellmans dynamic programming (DP) requires knowledge of transition
% probablities of the dynamic system from ones state to the next
%"""
%




The 8 DOF in our model produces vast amounts of space in our feature space and this has the consequence that we need proportional to $n^8$ datapoints to fill it to an specified density c.

For instance, we would need $4^8=65536$ samples just to make a grid with 4 samples in each DOF which in many cases (aspecially for hand labeled data). 

This phenomen of having hue searchspace in highdimensinonal space is fairly common and have the name "Curse of dimensionality".

One way to somewhat overcome this emptyness in space is to have a dynamic (active?) algorithm that adopts the sample density 
according to the models relative frequency in that area and this results in, for a given amount of samples, its more likely 
for an sampled model to occure in a more densly pre-sampled (pre-sampled?) area, in fact this method when doing it right (ideally) 
gives: given a set of samples the overall density for all the samples in the sampledatabase would be optimal) [proof for this will be given in <blabla>]
... (use the world hypotesis instead of sample, or perhaps sampled-hypotesis)
(below is perhaps somewhat redudant, but one can pick bits and pieces from both)
So having a bias towards trying out more plausible hypotesis for the models innerstate is better than 
doing a naive exhausted search tru the entire feature space, this could only be used if you have $\le$ 3 DOF as in for example finding straight edges with houghtransformation[ref].
...
One method that uses prior knowledge of the models PDF and a pretrained database containing knowledge on how to approximate the models PDF in the next timestep is particle filter which is an (instance?) of the ideal bayesian filter.



===================

\begin{definition}
    The render function $R$ takes an hypotesis $x$ and renders an image with an
    reassemblance on how a real whisker would have looked like having the same
    underlying model and parameters as $x$.
    \begin{equation}
        \begin{split}
            R : \HS &\rightarrow \IS\\
                x &\mapsto render(x)
        \end{split}
    \end{equation}
\end{definition}


\subsection{Sensory Cues}

The biggest problem with computer vision is that computers do not have
vision, but only a data input device in the form of a camera. 

<ref to machine vs biological comparison study>


\begin{definition}
    %%TODO cue function/transformation (cue says more doesnt it?) change all
    %%use
    The cue function $\phi_{cue}$ takes an image $I$ and renders(depicts?) one property of the
    image like intensity, edges or ridges.
    \begin{equation}
        \begin{split}
            \phi_{cue} : \IS &\rightarrow \IS\\
                I &\mapsto cue(I)
        \end{split}
    \end{equation}
    If the cue function for any cue is under consideration we simply denote that with $\phi$.
\end{definition}


<image showing the use of $\phi$>


\subsection{Particle filtering whisker movements}
Here we propose a way to model whiskers as a dynamic system.

\subsection{Kinematic whisker models}

In all our models we have separated the head from the whiskers since they have
such different kinematic properties and actually are attached to each other.


Assuming that a whisker has one fix point of origin<ref:analysis(this is not
the case really)> we have the boundary condition $whisker(0)=0$

"There are many metrics by which a model may be assessed." - Encyclopedia
(>model evaluation)

"
==Model selection
Model selection is the process of choosing an appropriate mathematical model
from a class of models.
" - Encyclopedia 
The class being all functions with f(0)=0,contioiuns(or stronger?) and finite number of parameters




The equation of elastic line ... [Grundläggande Hållfasthetslära - Hans Lundh
p94 (7.6)]

The force that comes from the head moving on the base of the whisker is just 
sucked up by the Boundaryvalues and it will still be valid assumptions for the
elasticline to hold.

Under just a few assumptions that the material is linear elastic and the
deformations are small we have the ...



\begin{theorem} %TODO svagare?
    \label{thm:response_max}
    Let $f$ be a positive Riemann function with compact support, then
    \begin{equation}
        \argmax{\bar{e}}
            \left(
            \sum\limits_{\Omega}{f(\bar{x})f(\bar{x}-\bar{e})}
            \right)
            =0
    \end{equation}
\end{theorem}
\begin{proof}
    Firstly define the window function as
    \begin{equation}
        W_a^b(x)=
        \begin{cases}
            0,~& x<a\\
            1,~& a=<x=<b\\
            0,~& x>b
        \end{cases}
    \end{equation}
    Multiplication
    \begin{equation}
        (W_a^bW_c^d)(x)=W_{\max(a,c)}^{\min(b,d)}(x)
    \end{equation}
    Translation
    \begin{equation}
        W_a^b(x-e)=W_{a+e}^{b+e}(x)
    \end{equation}
    Integration
    \begin{equation}
        \label{eq:int_window}
        \sum{W_a^b(x)dx}=\Theta(b-a)
    \end{equation}

    \begin{equation}
        \begin{array}{c}
            
            \argmax{e}\left(
                \sum
                    {
                        W_a^b(x)W_a^b(x-e)
                    }
                \right)
                =\\
            \argmax{e}\left(
                    \sum
                        {
                            W_a^b(x)W_
                                {
                                    a+e
                                }
                                ^
                                {
                                    b+e
                                }
                                (x)
                        }
                \right)
                =\\
            \argmax{e}\left(
                    \sum
                        {
                            W_
                                {
                                    \max(a,a+e)
                                }^
                                {
                                    \min(b,b+e)
                                }
                                (x)
                        }
                 \right)
                =\\
            \argmax{e}\left(
                    \Theta(\min(b,b+e)-\max(a,a+e))
                 \right)
                =\\
            %\left\{\text{
            %        max the min and min the max at the same time
            %        }
            %    \right\}=\\
            0
        \end{array}
      \end{equation}

    This trivially holds with superposition of windows, since all windows will scale and translate the same way. 
    With a finite support $e=0$ is the only solution. Additionaly this also hold in higher finite dimensions since we can just repeat the process for one dimension at a time.
    
    All riemann functions can be written as a superposition of windows like this
    \begin{equation}
        f(x)=\sum{c_iW_{a_i}^{b_i}(x)}
    \end{equation}

    $\therefore$ Each riemann function $f$ with finite support will have a $e=0\qed$

\end{proof}

=============================

\subsubsection{ Theoretical evalutaion (formal methods)}

============================

============== MODELS =================

One possible model is to borrow the model for beam under small 
deformations from the theory of strength of materials,
after all the whisker is a beam but we dont have small 
deformations at all but we assume that the model will approximatly hold ony way.


