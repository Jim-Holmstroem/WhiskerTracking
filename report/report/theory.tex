\newcommand{\xtmN}[4]{
  \left\{#1_{#2}^{#3}\right\}_{#2=1}^{#4}
}
\newcommand{\interval}[2]{
  \left[#1, #2\right]
}
\newcommand{\bel}[1]{
  bel\left(#1\right)
}

\subsection{The particle filter}
\subsubsection{Introduction}
The core of our tracking engine is a technique known as the \emph{particle filter}. The particle filter is a kind of Bayesian filtering where one uses discrete hypotheses to approximate continuous probability distributions. The main idea can be outlined as follows.

Suppose we have a system described by the state $x_t$ at time $t$. $x_t$ can be thought of as a vector of state parameters. Suppose that we know the previous state $x_{t-1}$ and have an observation $z_t$ of the system at the current time. In general, it is difficult to accurately determine $x_t$ from the observation alone, and the observation may suffer from interference. Therefore, we cannot directly read $x_t$ from $z_t$. However, we can estimate $x_t$ if we know the following things about the system:

\begin{itemize}
\item A probability function $p_t$, where $p_t\left(x_t | x_{t-1}\right)$ is the probability that the current state is $x_t$ if the previous state was $x_{t-1}$.
\item A probability function $q_t$, where $q_t\left(z_t | x_t\right)$ is the probability that we observe $z_t$ if the current state is $x_t$.
\end{itemize}

Using $p_t$ and knowing $x_{t-1}$, we can generate a set of hypotheses $\xtmN{\bar{x}}{t}{m}{N}$ for the current state $x_t$. Using $q_t$ and knowing $z_t$, we can evaluate how likely the hypotheses are. If it is likely to observe $z_t$ if the current state is $x_t$, then $x_t$ is probably a good estimate of the current state. With this information, we select the most probable hypotheses and let them be our estimate of the current system state.

The particle filter works recursively in two steps:

\begin{enumerate}
\item The \emph{sampling step} is the generation of the hypotheses, known as \emph{particles}, from the previous state $x_{t-1}$. The belief $\bel{x_{t-1}}$, see below, is used as an estimate for $x_{t-1}$. What makes this recursive is the fact that $\bel{x_t}$ is calculated using $\bel{x_{t-1}}$.

\item The \emph{resampling step} is the final selection of the most probable particles. After resampling the set $\bar{X_t} := \xtmN{\bar{x}}{t}{m}{N}$ we get the \emph{belief} $\bel{x_t}$, which often includes multiple copies of the most probable particles. This set is used as an estimate for $x_t$, and is used to estimate $x_{t+1}$.
\end{enumerate}

In the next section, we state the particle filter algorithm. Implementing the filter in itself is only a matter of implementing the stated pseudocode, and is not difficult. The difficult part is designing the probability functions $p_t$ and $q_t$ for the given system. In chapters TODO we propose a probabilistic implementation of these functions.


\subsubsection{Formal description of the particle filter}
Here we state the particle filter algorithm. Below we elaborate on what this actually does and why.
\begin{enumerate}
\item ParticleFilter($X_{t-1}, z_t$):
\item Let $\bar{X_t} = \emptyset$.
\item For $m:=1$ to $\left|X_{t-1}\right|$:
  \begin{enumerate}
  \item Draw $x_t^m$ with probability $p_t\left(x_t^m | x_{t-1}^m\right)$.
  \item Let $w_t^m := q_t\left(z_t | x_t^m\right)$.
  \item Add $(x_t^m, w_t^m)$ to $\bar{X_t}$.
  \end{enumerate}
\item Let $X_t := \emptyset$.
\item For $m:=1$ to $\left|X_{t-1}\right|$:
  \begin{enumerate}
    \item Draw $x_t^m$ with probability proportional to $w_t^m$
    \item Add $x_t^m$ to $X_t$.
  \end{enumerate}
\item Return $X_t$.
\end{enumerate}


We draw a set $\bar{X}_t = \xtmN{x}{m}{t}{N}$ of samples from $p_t$. These samples roughly represent a probability distribution for the current state $x_t$, but we have yet to consider our observation. Therefore, we will create a new probability distribution weighted by how probable the observation $z_t$ is. For each $x_t^m$, we let $w_t^m := q_t\left(z_t | x_t\right)$. This defines a discrete probability distribution where $x_t^m$ is assumed with a probability proportional to $w_t^m$. From this final distribution we again draw $N$ samples $X_t$, which will be our estimate of the current state. The elements $x_t^m$ are referred to as \emph{particles} and the set $X_t$ as the \emph{belief at time $t$}.

In our case, we do not really know $x_{t-1}$, rather we have an estimate $X_{t-1}$ of $N$ particles. In this case, we sample $x_t^m$ with probability $p_t\left(x_t^m | x_{t-1}^m\right)$.

\subsection{Particle filtering whisker movements}
Here we propose a way to model whiskers as a dynamic system.

\subsection{Kinematic whisker models}
What are our kinematic models?
