\newcommand{\xtmN}[4]{
  \left\{#1_{#2}^{#3}\right\}_{#2=1}^{#4}
}
\newcommand{\interval}[2]{
  \left[#1, #2\right]
}
\newcommand{\bel}[1]{
  bel\left(#1\right)
}

\subsection{The particle filter}
\subsubsection{Introduction}
The core of our tracking engine is a technique known as the \emph{particle filter}. The particle filter is a kind of Bayesian filtering where one uses discrete hypotheses to approximate continuous probability distributions. The main idea can be outlined as follows.

Suppose we have a system described by the state $x_t$ at time $t$. $x_t$ can be thought of as a vector of state parameters. Suppose that we know the previous state $x_{t-1}$ and have an observation $z_t$ of the system at the current time. In general, it is difficult to accurately determine $x_t$ from the observation alone, and the observation may suffer from interference. Therefore, we cannot directly read $x_t$ from $z_t$. However, we can estimate $x_t$ if we know the following things about the system:

\begin{itemize}
\item A probability function $p_t$, where $p_t\left(x_t | x_{t-1}\right)$ is the probability that the current state is $x_t$ if the previous state was $x_{t-1}$.
\item A probability function $q_t$, where $q_t\left(z_t | x_t\right)$ is the probability that we observe $z_t$ if the current state is $x_t$.
\end{itemize}

Using $p_t$ and knowing $x_{t-1}$, we can generate a set of hypotheses $\xtmN{\bar{x}}{t}{m}{N}$ for the current state $x_t$. Using $q_t$ and knowing $z_t$, we can evaluate how likely the hypotheses are. If it is likely to observe $z_t$ if the current state is $x_t$, then $x_t$ is probably a good estimate of the current state. With this information, we select the most probable hypotheses and let them be our estimate of the current system state.

The particle filter works recursively in two steps:

\begin{enumerate}
\item The \emph{sampling step} is the generation of the hypotheses, known as \emph{particles}, from the previous state $x_{t-1}$. The belief $\bel{x_{t-1}}$, see below, is used as an estimate for $x_{t-1}$. What makes this recursive is the fact that $\bel{x_t}$ is calculated using $\bel{x_{t-1}}$.

\item The \emph{resampling step} is the final selection of the most probable particles. After resampling the set $\bar{X_t} := \xtmN{\bar{x}}{t}{m}{N}$ we get the \emph{belief} $\bel{x_t}$, which often includes multiple copies of the most probable particles. This set is used as an estimate for $x_t$, and is used to estimate $x_{t+1}$.
\end{enumerate}

In the next section, we state the particle filter algorithm. Implementing the filter in itself is only a matter of implementing the stated pseudocode, and is not difficult. The difficult part is designing the probability functions $p_t$ and $q_t$ for the given system. In chapters TODO we propose a probabilistic implementation of these functions.


\subsubsection{Formal description of the particle filter}
Here we state the particle filter algorithm. Below we elaborate on what this actually does and why.
%TODO pseudo algorithm for the 
%Sample instead of draw

\begin{codebox}
\Procname{$\proc{Distribution-Sample}(X_t,p)$}
\li \ForEach $\id{x_t}$ \In $\id{X_t}$
\li     \Do
            $sample \id{x_{t+1}} \sim p\left(x_{t+1}|x_t\right)$
        \End
\li \Return $\id{X_{t+1}}$
\end{codebox}
\begin{codebox}
\Procname{$\proc{Importance}(X,q,z)$}
\li \ForEach $\id{x}$ \In $\id{X}$ 
\li     \Do
            $w \gets q\left(z|x\right)$
        \End
\li \Return $W$
\end{codebox}
\begin{codebox}
\Procname{$\proc{Weighted-Sample}(X,W)$}
\li \ForEach $x$ \In $X$
\li     \Do
            $sample ~ x' \propto W$   
        \End
\li \Return $X'$
\end{codebox}
\begin{codebox}
\Procname{$\proc{Particle-Filter} (X_{t-1},z_t)$}
\li $X_t \gets \proc{Distribution-Sample}(X_{t-1},p)$
\li $W_t \gets \proc{Importance}(X_{t-1},q,z_t)$
\li $X_t \gets \proc{Weighted-Sample}(X_t,W_t)$
\li \Return $X_t$
\end{codebox}

\begin{enumerate}
\item ParticleFilter($X_{t-1}, z_t$):
\item Let $\bar{X_t} = \emptyset$.
\item For $m:=1$ to $\left|X_{t-1}\right|$:
  \begin{enumerate}
  \item Draw $x_t^m$ with probability $p_t\left(x_t^m | x_{t-1}^m\right)$.
  \item Let $w_t^m := q_t\left(z_t | x_t^m\right)$.
  \item Add $(x_t^m, w_t^m)$ to $\bar{X_t}$.
  \end{enumerate}
\item Let $X_t := \emptyset$.
\item For $m:=1$ to $\left|X_{t-1}\right|$:
  \begin{enumerate}
    \item Draw $x_t^m$ with probability proportional to $w_t^m$
    \item Add $x_t^m$ to $X_t$.
  \end{enumerate}
\item Return $X_t$.
\end{enumerate}


We draw a set $\bar{X}_t = \xtmN{x}{m}{t}{N}$ of samples from $p_t$. These samples roughly represent a probability distribution for the current state $x_t$, but we have yet to consider our observation. Therefore, we will create a new probability distribution weighted by how probable the observation $z_t$ is. For each $x_t^m$, we let $w_t^m := q_t\left(z_t | x_t\right)$. This defines a discrete probability distribution where $x_t^m$ is assumed with a probability proportional to $w_t^m$. From this final distribution we again draw $N$ samples $X_t$, which will be our estimate of the current state. The elements $x_t^m$ are referred to as \emph{particles} and the set $X_t$ as the \emph{belief at time $t$}.

In our case, we do not really know $x_{t-1}$, rather we have an estimate $X_{t-1}$ of $N$ particles. In this case, we sample $x_t^m$ with probability $p_t\left(x_t^m | x_{t-1}^m\right)$.



==Curse of dimensinallity
{problem}
The 8 DOF in our model produces vast amounts of space in our featurespace and this has the consequence that we need \proportianal n^8 datapoints to fill it to an specified density c.
As an example, we would need 4^8=65536 samples just to make a grid with 4 samples in each DOF which in many cases (aspecially for hand labeled data). 
{classification}
This phenomen of having hue searchspace in highdimensinonal space is fairly common and have the name "Curse of dimensionallity" http://en.wikipedia.org/wiki/Curse_of_dimensionality
{solution}
One way to somewhat overcome this emptyness in space is to have a dynamic (active?) algorithm that adopts the sample density 
according to the models relative frequency in that area and this results in, for a given amount of samples, its more likely 
for an sampled model to occure in a more densly pre-sampled (pre-sampled?) area, in fact this method when doing it right (ideally) 
gives: given a set of samples the overall density for all the samples in the sampledatabase would be optimal) [proof for this will be given in <blabla>]
... (use the world hypotesis instead of sample, or perhaps sampled-hypotesis)
(below is perhaps somewhat redudant, but one can pick bits and pieces from both)
So having a bias towards trying out more plausible hypotesis for the models innerstate is better than 
doing a naive exhausted search tru the entire featurespace, this could only be used if you have \le 3 DOF as in for example finding straight edges with houghtransformation[ref].
...
One method that uses prior knowledge of the models PDF and a pretrained database containing knowledge on how to approximate the models PDF in the next timestep is particle filter which is an (instance?) of the ideal bayesian filter.






\subsection{Particle filtering whisker movements}
Here we propose a way to model whiskers as a dynamic system.

\subsection{Kinematic whisker models}

In all our models we have separated the head from the whiskers since they have
such different kinematic properties and actually are attached to each other.

==The first model
Assumptions:
>Material is linear elastic
>Small deformations (probably the biggest assumption, which dont actually hold
in our case)

The equation of elastic line ... [Grundläggande Hållfasthetslära - Hans Lundh
p94 (7.6)]

The force that comes from the head moving on the base of the whisker is just 
sucked up by the Boundaryvalues and it will still be valid assumptions for the
elasticline to hold.

Under just a few assumptions that the material is linear elastic and the
deformations are small we have the ...


============== MODELS =================

One possible model is to borrow the model for beam under small 
deformations from the theory of strength of materials,
after all the whisker is a beam but we dont have small 
deformations at all but we assume that the model will approximatly hold ony way.


