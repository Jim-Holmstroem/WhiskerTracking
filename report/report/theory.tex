\subsection{Introduction}
The core of tracking engine used is a technique known as the \emph{particle filter}. The particle filter is a kind of Bayesian filtering where one uses discrete hypotheses, also known as \emph{particles} to approximate continuous probability distributions \cite{ProbRob}. It builds upon the theory of \emph{Markov Processes} and the \emph{Hidden Markov Model}.

\subsection{Markov processes and the Hidden Markov Model}
A Markov process is a special case of a stochastic process. For a Markov process, the next state depends only on the present state and not on past states. For this reason, a Markov process is often said to be ``forgetful''.

\begin{figure}
  \centering
  \includegraphics[width=0.8\textwidth]{hmm-graph.pdf}
  \caption{Schematic image of a Hidden Markov Model.}
  \label{fig:hmm-graph}
\end{figure}

\subsection{Hidden Markov Processes}

Our solution is based on \emph{Markov processes}, which are 
An example of a Markov process is that of throwing dice and summing the results: the possible states (sums) after the next throw depends only on the current state.

In mathematical terms, a Markov process satisfies the following:
\begin{equation}
 p\left(Z_{n+1}|Z_n \wedge Z_{n-1} \wedge Z_{n-2} \wedge \dots \wedge Z_0\right) = p\left(Z_{n+1}|Z_n\right),
\end{equation}
where $Z_n$ is the system's \emph{state} after step $n$ and $p\left(Z_{n+1}|Z_n, Z_{n-1}, Z_{n-2}, \dots, Z_0\right)$ is the probability that the system will have state $Z_{n+1}$ in the next step, given that the previous states where $Z_n, Z_{n-1}, Z_{n-2}, \dots, Z_0$.

A \emph{hidden Markov model} (HMM) describes a Markov process where one cannot measure the state $Z$ of the system directly - it is ``hidden''\cite{EncyclopediaMachineLearning}. Instead we obtain an \emph{observation} $I$\footnote{In this project, the observation is a grayscale \emph{image}, which is why we use the symbol $I$.}  of the state. This \emph{perception} is generally non-deterministic, so we need to denote it as $p(I_n|Z_n)$ which is the probability that we will observe $I_n$ if the current state of the system is $Z_n$.

\subsection{The particle filter}

\subsubsection{Introduction}
The core of tracking engine used is a technique known as the \emph{particle filter}. The particle filter is a kind of Bayesian filtering where one uses discrete hypotheses to approximate continuous probability distributions \cite{ProbRob}. The main idea can be outlined as follows.

Suppose we have a system described by the state $x_t$ at time $t$. $x_t$ can be thought of as a vector of state parameters. Suppose that we know the previous state $x_{t-1}$ and have an observation $z_t$ of the system at the current time. In general, it is difficult to accurately determine $x_t$ from the observation alone, and the observation may suffer from interference. Therefore, we cannot directly read $x_t$ from $z_t$. However, we can estimate $x_t$ if we know the following things about the system:

\begin{itemize}
\item A probability function $p$, where $p\left(x_t | x_{t-1}\right)$ is the probability that the current state is $x_t$ if the previous state was $x_{t-1}$.
\item A probability function $q$, where $q\left(z_t | x_t\right)$ is the probability that we observe $z_t$ if the current state is $x_t$.
\end{itemize}

Using $p$ and knowing $x_{t-1}$, we can generate a set of hypotheses $\xtmN{\bar{x}}{t}{m}{N}$ for the current state $x_t$. Using $q$ and knowing $z_t$, we can evaluate how likely the hypotheses are. If it is likely to observe $z_t$ if the current state is $x_t$, then $x_t$ is probably a good estimate of the current state. With this information, we select the most probable hypotheses and let them be our estimate of the current system state.






The main idea can be outlined as follows.

Suppose we have a system described by the state $x_t$ at time $t$. $x_t$ can be thought of as a vector of state parameters. Suppose that we know the previous state $x_{t-1}$ and have an observation $z_t$ of the system at the current time. In general, it is difficult to accurately determine $x_t$ from the observation alone, and the observation may suffer from interference. Therefore, we cannot directly read $x_t$ from $z_t$. However, we can estimate $x_t$ if we know the following things about the system:

\begin{itemize}
\item A probability function $p$, where $p\left(x_t | x_{t-1}\right)$ is the probability that the current state is $x_t$ if the previous state was $x_{t-1}$.
\item A probability function $q$, where $q\left(z_t | x_t\right)$ is the probability that we observe $z_t$ if the current state is $x_t$.
\end{itemize}

Using $p$ and knowing $x_{t-1}$, we can generate a set of hypotheses $\xtmN{\bar{x}}{t}{m}{N}$ for the current state $x_t$. Using $q$ and knowing $z_t$, we can evaluate how likely the hypotheses are. If it is likely to observe $z_t$ if the current state is $x_t$, then $x_t$ is probably a good estimate of the current state. With this information, we select the most probable hypotheses and let them be our estimate of the current system state.

The particle filter works recursively in two steps:

\begin{enumerate}
\item The \emph{sampling step} is the generation of the hypotheses, known as \emph{particles}, from the previous state $x_{t-1}$. The belief $\bel{x_{t-1}}$, see below, is used as an estimate for $x_{t-1}$. What makes this recursive is the fact that $\bel{x_t}$ is calculated using $\bel{x_{t-1}}$.

\item The \emph{resampling step} is the final selection of the most probable particles. After resampling the set $\bar{X_t} := \xtmN{\bar{x}}{t}{m}{N}$ we get the \emph{belief} $\bel{x_t}$, which often includes multiple copies of the most probable particles. This set is used as an estimate for $x_t$, and is used to estimate $x_{t+1}$.
\end{enumerate}

In the next section, the particle filter algorithm will be stated. Implementing the filter in itself is only a matter of implementing the stated pseudocode, and is not difficult. The difficult part is designing the probability functions $p$ and $q$ for the given system. A probabilistic implementation of these functions is proposed in chapters TODO.


\subsection{Hidden Markov Processes}

Our solution is based on \emph{Markov processes}, which are a special case of stochastic processes. For a Markov process, the next state depends only on the present state and not on past states.

An example of a Markov process is that of throwing dice and summing the results: the possible states (sums) after the next throw depends only on the current state.

In mathematical terms, a Markov process satisfies the following:
\begin{equation}
 p\left(Z_{n+1}|Z_n \wedge Z_{n-1} \wedge Z_{n-2} \wedge \dots \wedge Z_0\right) = p\left(Z_{n+1}|Z_n\right),
\end{equation}
where $Z_n$ is the system's \emph{state} after step $n$ and $p\left(Z_{n+1}|Z_n, Z_{n-1}, Z_{n-2}, \dots, Z_0\right)$ is the probability that the system will have state $Z_{n+1}$ in the next step, given that the previous states where $Z_n, Z_{n-1}, Z_{n-2}, \dots, Z_0$.

A \emph{hidden Markov model} (HMM) describes a Markov process where one cannot measure the state $Z$ of the system directly - it is ``hidden''\cite{EncyclopediaMachineLearning}. Instead we obtain an \emph{observation} $I$\footnote{In this project, the observation is a grayscale \emph{image}, which is why we use the symbol $I$.}  of the state. This \emph{perception} is generally non-deterministic, so we need to denote it as $p(I_n|Z_n)$ which is the probability that we will observe $I_n$ if the current state of the system is $Z_n$.

\subsection{The particle filter}

\subsubsection{Introduction}
The core of tracking engine used is a technique known as the \emph{particle filter}. The particle filter is a kind of Bayesian filtering where one uses discrete hypotheses to approximate continuous probability distributions \cite{ProbRob}. The main idea can be outlined as follows.

Suppose we have a system described by the state $x_t$ at time $t$. $x_t$ can be thought of as a vector of state parameters. Suppose that we know the previous state $x_{t-1}$ and have an observation $z_t$ of the system at the current time. In general, it is difficult to accurately determine $x_t$ from the observation alone, and the observation may suffer from interference. Therefore, we cannot directly read $x_t$ from $z_t$. However, we can estimate $x_t$ if we know the following things about the system:

\begin{itemize}
\item A probability function $p$, where $p\left(x_t | x_{t-1}\right)$ is the probability that the current state is $x_t$ if the previous state was $x_{t-1}$.
\item A probability function $q$, where $q\left(z_t | x_t\right)$ is the probability that we observe $z_t$ if the current state is $x_t$.
\end{itemize}

Using $p$ and knowing $x_{t-1}$, we can generate a set of hypotheses $\xtmN{\bar{x}}{t}{m}{N}$ for the current state $x_t$. Using $q$ and knowing $z_t$, we can evaluate how likely the hypotheses are. If it is likely to observe $z_t$ if the current state is $x_t$, then $x_t$ is probably a good estimate of the current state. With this information, we select the most probable hypotheses and let them be our estimate of the current system state.

The particle filter works recursively in two steps:

\begin{enumerate}
\item The \emph{sampling step} is the generation of the hypotheses, known as \emph{particles}, from the previous state $x_{t-1}$. The belief $\bel{x_{t-1}}$, see below, is used as an estimate for $x_{t-1}$. What makes this recursive is the fact that $\bel{x_t}$ is calculated using $\bel{x_{t-1}}$.

\item The \emph{resampling step} is the final selection of the most probable particles. After resampling the set $\bar{X_t} := \xtmN{\bar{x}}{t}{m}{N}$ we get the \emph{belief} $\bel{x_t}$, which often includes multiple copies of the most probable particles. This set is used as an estimate for $x_t$, and is used to estimate $x_{t+1}$.
\end{enumerate}

In the next section, the particle filter algorithm will be stated. Implementing the filter in itself is only a matter of implementing the stated pseudocode, and is not difficult. The difficult part is designing the probability functions $p$ and $q$ for the given system. A probabilistic implementation of these functions is proposed in chapters TODO.


\subsubsection{Formal description of the particle filter}
Here the particle filter algorithm is stated. An elaboration on what this actually does and why is offered below.
%TODO pseudo algorithm for the 
%Sample instead of draw

\begin{codebox}
\Procname{$\proc{Distribution-Sample}(X_t,p)$}
\li \ForEach $\id{x_t}$ \In $\id{X_t}$
\li     \Do
            $sample \id{x_{t+1}} \sim p\left(x_{t+1}|x_t\right)$
        \End
\li \Return $\id{X_{t+1}}$
\end{codebox}
\begin{codebox}
\Procname{$\proc{Importance}(X,q,z)$}
\li \ForEach $\id{x}$ \In $\id{X}$ 
\li     \Do
            $w \gets q\left(z|x\right)$
        \End
\li \Return $W$
\end{codebox}
\begin{codebox}
\Procname{$\proc{Weighted-Sample}(X,W)$}
\li \ForEach $x$ \In $X$
\li     \Do
            $sample ~ x' \propto W$   
        \End
\li \Return $X'$
\end{codebox}
\begin{codebox}
\Procname{$\proc{Particle-Filter} (X_{t-1},z_t)$}
\li $X_t \gets \proc{Distribution-Sample}(X_{t-1},p)$
\li $W_t \gets \proc{Importance}(X_{t-1},q,z_t)$
\li $X_t \gets \proc{Weighted-Sample}(X_t,W_t)$
\li \Return $X_t$
\end{codebox}

%\begin{codebox}
%\Procname{$\proc{Particle-Filter} (X_{t-1}, z_t)$}
%\li Let $\bar{X_t} = \emptyset$.
%\li For $m:=1$ to $\left|X_{t-1}\right|$:
%\li \Do Draw $x_t^m$ with probability $p\left(x_t^m | x_{t-1}^m\right)$.
%\li Let $w_t^m := q\left(z_t | x_t^m\right)$.
%\li Add $(x_t^m, w_t^m)$ to $\bar{X_t}$.
%\End
%\li Let $X_t := \emptyset$.
%\li For $m:=1$ to $\left|X_{t-1}\right|$:
%\li \Do
%    Draw $x_t^m$ with probability proportional to $w_t^m$
%    \li Add $x_t^m$ to $X_t$.
%    \End
%\li Return $X_t$.
%\end{codebox}


We draw a set $\bar{X}_t = \xtmN{x}{m}{t}{N}$ of samples from $p$. These samples roughly represent a probability distribution for the current state $x_t$, but we have yet to consider our observation. Therefore, we will create a new probability distribution weighted by how probable the observation $z_t$ is. For each $x_t^m$, we let $w_t^m := q\left(z_t | x_t\right)$. This defines a discrete probability distribution where $x_t^m$ is assumed with a probability proportional to $w_t^m$. From this final distribution we again draw $N$ samples $X_t$, which will be our estimate of the current state. The elements $x_t^m$ are referred to as \emph{particles} and the set $X_t$ as the \emph{belief at time $t$}.

In this thesis, $x_{t-1}$ is not known. Rather $x_{t-1}$ is estimated with a set $X_{t-1}$ of $N$ particles. In this case, $x_t^m$ is sampled with probability $p\left(x_t^m | x_{t-1}^m\right)$.



\subsubsection{Curse of Dimensonality}
One phenomena that becomes apperant in higher dimensional space is the Curse of dimensionality<2xref>.
The problem is that the search-volume grows exponentialy with
the number of dimensions. It originates from the fact that to have 
a sample density of $C$ we need $\Ordo{C^n}$ samples for $n$-dimensional data.
The first consequence of this is that in order to approximate some
function defined in higher dimensional space needs order of magnitudes more samples.
The other drawback with high dimensional space is the large ``boundaries'' 
of the sample-set compared to lower dimensional space which results in orders of
magnitude higher chance for an point one want to approximate to fall outside
the sample-set and needs to be extrapolated instead of the better alternative of interpolation.


\begin{example}
The effect on $128$ samples transformed $1D\rightarrow2D\rightarrow3D$
\begin{figure}
    \begin{tabular}{rcl}
        \includegraphics[scale=0.3,trim=4cm 4cm 4cm 4cm]{1D.pdf}&
        \includegraphics[scale=0.3,trim=4cm 4cm 4cm 4cm]{2D.pdf}&
        \includegraphics[scale=0.3,trim=4cm 4cm 4cm 4cm]{3D.pdf}
    \end{tabular}
    \caption{Plot of $128$ scattered samples in $1$ (left), $2$ (center) and $3$ (right) dimensions.}
\end{figure}
\end{example}

\begin{example}
    For a model with $16$ degrees of freedom
    \footnote{The number of adjustable parameters in the model} (DOF) 
    one needs $10^16=10$quadrillion datapoints to acquire a density
    of $10$ samples per volume unit. We would need millions of gigabytes only to save the result.
\end{example}

\begin{example}
    Exhausted search of $3$ dimensional space can be done as in for example in houghtransform.
\end{example}


%This phenomen of having hue searchspace in highdimensinonal space is fairly common and have the name "Curse of dimensionality".

%One way to somewhat overcome this emptyness in space is to have a dynamic (active?) algorithm that adopts the sample density 
%according to the models relative frequency in that area and this results in, for a given amount of samples, its more likely 
%for an sampled model to occure in a more densly pre-sampled (pre-sampled?) area, in fact this method when doing it right (ideally) 
%gives: given a set of samples the overall density for all the samples in the sampledatabase would be optimal) [proof for this will be given in <blabla>]
%... (use the world hypotesis instead of sample, or perhaps sampled-hypotesis)
%(below is perhaps somewhat redudant, but one can pick bits and pieces from both)
%So having a bias towards trying out more plausible hypotesis for the models innerstate is better than 
%doing a naive exhausted search tru the entire feature space, this could only be used if you have $\le$ 3 DOF as in for example finding straight edges with houghtransformation[ref].
%...
%One method that uses prior knowledge of the models PDF and a pretrained database containing knowledge on how to approximate the models PDF in the next timestep is particle filter which is an (instance?) of the ideal bayesian filter.


% """
%The "Curse of dimensionality", is a term coined by Bellman to describe the
%problem caused by the exponential increase in volume associated with adding
%extra dimensions to a (mathematical) space. One implication of the curse of
%dimensionality is that some methods for numerical solution of the Bellman
%equation require vastly more computer time when there are more state variables
%in the value function.

%For example, 100 evenly-spaced sample points suffice to sample a unit interval
%with no more than 0.01 distance between points; an equivalent sampling of a
%10-dimensional unit hypercube with a lattice with a spacing of 0.01 between
%adjacent points would require 1020 sample points: thus, in some sense, the
%10-dimensional hypercube can be said to be a factor of 1018 "larger" than the
%unit interval. (Adapted from an example by R. E. Bellman, see below.)
% """

%from:
%"R. Bellman, Adaptive control Processes, p.94, Princeton University Press, NJ,
%1961."

%“In view of all that we have said in the foregoing
%sections, the many obstacles we appear to have
%surmounted. What casts the pall over our victory
%celebration? It is the curse of dimensionality, a
%malediction that has plagued the scientist from
%earliest days.”

% """
%
% Number of states grows exponentially in n (assuming fixed number of
% discretization levels per coordinate)
%
% """

%
%"""
% One solution on how to make the effects of the curse of dimensionality is to
% make a directed search..
%"""
%

%
%"""
% Bellmans dynamic programming (DP) requires knowledge of transition
% probablities of the dynamic system from ones state to the next
%"""
%






===================

\begin{definition}
    The render function $R$ takes an hypotesis $x$ and renders an image with an
    reassemblance on how a real whisker would have looked like having the same
    underlying model and parameters as $x$.
    \begin{equation}
        \begin{split}
            R : \HS &\rightarrow \IS\\
                x &\mapsto render(x)
        \end{split}
    \end{equation}
\end{definition}


\subsection{Sensory Cues}

The biggest problem with computer vision is that computers do not have
vision, but only a data input device in the form of a camera. 

<ref to machine vs biological comparison study>


\begin{definition}
    %%TODO cue function/transformation (cue says more doesnt it?) change all
    %%use
    The cue function $\phi_{cue}$ takes an image $I$ and renders(depicts?) one property of the
    image like intensity, edges or ridges.
    \begin{equation}
        \begin{split}
            \phi_{cue} : \IS &\rightarrow \IS\\
                I &\mapsto cue(I)
        \end{split}
    \end{equation}
    If the cue function for any cue is under consideration we simply denote that with $\phi$.
\end{definition}


<image showing the use of $\phi$>


\subsection{Particle filtering whisker movements}
Here we propose a way to model whiskers as a dynamic system.

\subsection{Kinematic whisker models}

In all our models we have separated the head from the whiskers since they have
such different kinematic properties and actually are attached to each other.


\section{Model}

%"
%==Model selection
%Model selection is the process of choosing an appropriate mathematical model
%from a class of models.
%" - Encyclopedia 

A whisker can be modeled as a function:

% In the following chapter we will formalize the whisker and discuss
% our model selection\cite{EncylopediaMachineLearning} process.

% First of we formalize the whisker by the definition

\begin{definition}
    Let 
    \begin{equation}
    \begin{split}
        \text{whisker} : \RR^+ &\rightarrow \RR^2 \\
                  \omega&\mapsto \text{whisker}(\omega)
    \end{split}
    \end{equation}
\end{definition}
where $\omega$ is a coordinate along the length of the whisker and
$\text{whisker}(\omega)$ is a point in the $x$-$y$ plane.

The tracking problem is then to find a function $\text{whisker}^*$
that approximates $\text{whisker}$. The function class used in a
model must satisfy the following conditions:

\begin{enumerate}
\item It must be able to approximate the whisker sufficiently
  well.

  \begin{example}
    A straight line will not suffice since the $\text{whisker}$ is
    generally curved and straight lines can not fill that partition of
    the space.
  \end{example}
  
\item The functions must be $C_1$ and be possible to represent with a
  finite number of parameters.
\end{enumerate}

With this in mind, two classes of functions immediately appear as
candidates:
\begin{itemize}
\item Polynomials $\sum_{i=0}^{n} a_i\omega^i$
\item Fourier series $\sum_{k=0}^{n} a_k\sin(\frac{k\pi\omega}{L}) +
  b_k\cos(\frac{k\pi\omega}{L})$
\end{itemize}

A brief analysis of both will follow, after a discussion of
simplifications and difference measures..

\subsection{Simplifications}
One simplification one might make is to assume that the root of the
whisker is fixed in some point on the snout. This means that we can
let the whisker function be defined in a head-fixed coordinate system
for each whisker, with the root of the whisker at the origin. This
gives us the boundary condition
\begin{equation}
    \label{eq:bv_root}
    \text{whisker}^*(0)=\bar{0}.
\end{equation}

Manual inspection of whisker videos suggests that this is not the case
for real whiskers. However, there seems to be some point within the
snouts that stays approximately still and can be regarded as the root
of the whisker. A better model could take this into account, but that
will not be covered in this thesis.

The thickness of a whisker is not constant, but decreases as the
distance from the head increases. A simple model for this is to define
the whisker thickness as

\begin{equation}
    d(\omega) = \begin{cases}
        D-\frac{D\omega}{L},~& \omega<L\\
        0,~& \omega\ge L
    \end{cases}
\end{equation}

where $D>0$ is the thickness at the root and $L>0$ is the total length of the
whisker.

\subsection{Difference measure}
The standard way to quantify distance between functions defined on an
interval $\interval{a}{b}$ is to use the norm in the
$\Lp[2](\interval{a}{b}, 1)$ Hilbert space. In this thesis, the norms
for $p = 2, 4, 8$ will be used, and the impact of the choice of $p$
will be investigated. This means that condition 1 above says the model
must be such that the $\Lp$ distance $\Lpnorm{\text{whisker} -
  \text{whisker*}}$ can be made sufficiently small.

\subsection{Polynomial $\Spline{\omega}$}
The first and simplest candidate is the polynomials. Manual tests in
MATLAB indicate that three terms are enough to approximate a whisker
well enough that the difference is not visible to the naked eye. A
whisker function can therefore be modeled as a third degree polynomial
$\Spline{\omega}$, also known as a \emph{spline}.

These define parameterized curves in the $xy$ plane as
\begin{equation}
  (\omega,\Spline{\omega})
\end{equation}

This choice can be justified by comparing with the theory of beams
under small deformations in solid mechanics. After all, a whisker is
not too different from a beam. The two main assumptions for this to
hold is that deformations are small and that the effects of motion on
whisker deformation are negligible. \cite{Hallfasthet} This may not
quite be the case, but a spline is still capable of approximating the
momentaneous shape of a whisker well enough.

\subsection{Sine series $\sum a_k\sin \left(\frac{k\pi\omega}{L}\right)$}
Another promising candidate is the Fourier series. Considering
\eqref{eq:bv_root}, the Fourier series is reduced to a sine series
$\sum a_k\sin \left(\frac{k\pi\omega}{L}\right)$. However, such a
series will always be zero at $\omega = L$, and is therefore not a
reasonable model for whiskers. To address this, one could instead use
quarter-periods: $\sum a_k \sin\left( \frac{k\pi\omega}{2L}\right)$,
but at the cost of losing the orthogonality property of the sine
basis. Manual tests in MATLAB indicate that such a series to third
order performs approximately the same as a spline.

These define parameterized curves in the $xy$ plane as
\begin{equation}
    (\omega,\sum{a_n\sin (\frac{2\pi n}{L}\omega)})
\end{equation}


The choice of a sine series can be justified by comparing with the
theory of stiff strings in analytical mechanics. The movement equation
for a stiff string is a partial differential equation containing the
second and fourth spatial derivatives.\footnote{Source: Solving
  exercise 7.10 of \cite{VarKalk} using variational calculus.} A
classic separation of variables solution to the equation would be a
sine series for the spatial part.

\subsection{Theoretical evaluation}

It is hard to theoretically justify the choice of $\text{whisker}^*$
model since we do not have an analytic model for the $\text{whisker}$.

\begin{figure}
  \centering
  \includegraphics[width=0.45\textwidth]{rat-splines.png}
  \includegraphics[width=0.45\textwidth]{rat-sines.png}
  \caption{Comparison of whisker models. Left: Splines, Right: Sine series}
  \label{fig:model-comparison}
\end{figure}

For this thesis, the spline model was used in the tracking engine. The
main reason for this is that it is slightly simpler than the sine
model. It also is rather easy to get an intuitive grasp of how the
parameters affect the curve shape - the $\omega^3$ term mostly affects
the tip of the whisker while the $\omega$ term affects the overall
orientation.


The equation of elastic line ... [Grundläggande Hållfasthetslära - Hans Lundh
p94 (7.6)]

The force that comes from the head moving on the base of the whisker is just 
sucked up by the Boundaryvalues and it will still be valid assumptions for the
elasticline to hold.

Under just a few assumptions that the material is linear elastic and the
deformations are small we have the ...



\begin{theorem} %TODO svagare?
    \label{thm:response_max}
    Let $f$ be a positive Riemann function with compact support, then
    \begin{equation}
        \argmax{\bar{e}}
            \left(
            \sum\limits_{\Omega}{f(\bar{x})f(\bar{x}-\bar{e})}
            \right)
            =0
    \end{equation}
\end{theorem}
\begin{proof}
    Firstly define the window function as
    \begin{equation}
        W_a^b(x)=
        \begin{cases}
            0,~& x<a\\
            1,~& a=<x=<b\\
            0,~& x>b
        \end{cases}
    \end{equation}
    Multiplication
    \begin{equation}
        (W_a^bW_c^d)(x)=W_{\max(a,c)}^{\min(b,d)}(x)
    \end{equation}
    Translation
    \begin{equation}
        W_a^b(x-e)=W_{a+e}^{b+e}(x)
    \end{equation}
    Integration
    \begin{equation}
        \label{eq:int_window}
        \sum{W_a^b(x)dx}=\Theta(b-a)
    \end{equation}

    \begin{equation}
        \begin{array}{c}
            
            \argmax{e}\left(
                \sum
                    {
                        W_a^b(x)W_a^b(x-e)
                    }
                \right)
                =\\
            \argmax{e}\left(
                    \sum
                        {
                            W_a^b(x)W_
                                {
                                    a+e
                                }
                                ^
                                {
                                    b+e
                                }
                                (x)
                        }
                \right)
                =\\
            \argmax{e}\left(
                    \sum
                        {
                            W_
                                {
                                    \max(a,a+e)
                                }^
                                {
                                    \min(b,b+e)
                                }
                                (x)
                        }
                 \right)
                =\\
            \argmax{e}\left(
                    \Theta(\min(b,b+e)-\max(a,a+e))
                 \right)
                =\\
            %\left\{\text{
            %        max the min and min the max at the same time
            %        }
            %    \right\}=\\
            0
        \end{array}
      \end{equation}

    This trivially holds with superposition of windows, since all windows will scale and translate the same way. 
    With a finite support $e=0$ is the only solution. Additionaly this also hold in higher finite dimensions since we can just repeat the process for one dimension at a time.
    
    All riemann functions can be written as a superposition of windows like this
    \begin{equation}
        f(x)=\sum{c_iW_{a_i}^{b_i}(x)}
    \end{equation}

    $\therefore$ Each riemann function $f$ with finite support will have a $e=0\qed$

\end{proof}

=============================

\subsubsection{ Theoretical evalutaion (formal methods)}

============================

============== MODELS =================

One possible model is to borrow the model for beam under small 
deformations from the theory of strength of materials,
after all the whisker is a beam but we dont have small 
deformations at all but we assume that the model will approximatly hold ony way.


