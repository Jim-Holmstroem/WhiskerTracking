\section*{Bakgrund}

Det finns generellt sett två olika typer av metoder för att simulera ljud i datorspel idag\cite{funkhouser}. Det ena sättet är att använda numeriska lösningar av den akustiska vågekvationen, t.ex. finita element metoder, det andra är att högfrekvensapproximera ljudet hos mottagaren baserat på att följa strålgången från källa till mottagare. Det stora problemet är att all ljudsimulering, likt grafiksimulering, ska ske i realtid. Därför finns en tillåten beräkningstid som indirekt reglerar hur kostsam simuleringen får vara. Att simulera fenomen såsom reflektion, diffraktion eller håligheter under denna begränsning är en utmaning.\\* 
Den senaste forskningen har bland annat gått ut på att få ner beräkningstiden för strålgångsmetoder, samt utveckla nya strålgångsmetoder och utvidga dessa för att lösa metodens brister. En huvudsaklig anledning är att GPU:erna (Graphics Processing Unit) i dagens datorer gör liknande operationer för ljus och grafik. Därför kan mycket prestanda vinnas om man lyckas att effektivt använda delar av grafiksimuleringen till att samtidigt behandla ljudutbredningen\cite{sjoberg}. Detta gäller dock inte bara strålgång, utan samma typ av lösning skulle gynna andra metoder med sämre komplexitet. GPU:er kan utföra fler operationer parallellt än vad en CPU (Central Processing Unit) kan. Om man skulle kunna avlasta CPU:n, som är den processor som utför bland annat ljudsimuleringar, kan man tjäna mycket beräkningstid, speciellt i realtidsberäkningar\cite{CPUcoGPU}. GPU:ns parallella arbetssätt gör att den är lämplig för att utföra FFT, Fast Fourier Transform, en algorim som kan användas för snabb beräkning av linjära partiella differentialekvationer. Detta är våran motivering till studiet av ljudsimulering baserat på den linjära akustiska vågekvationen. Detta skulle ge oss en realistisk modell för ljud i datorspel.\\*
\newline
Den linjära akustiska vågekvationen beskriver utveklingen av partikelpositionen $u$ eller det akustiska trycket $\rho$ som funktion av positionen och tiden. Ekvationen lyder för positionen
\begin{equation}
u_{tt} - v^{2}\Delta u = f(x,y,z,t),
\end{equation}
där v är ljudhastigheten i det aktuella mediet och $f(x,y,z,t)$ är en ljudkälla.\\*
I datorspel så rör sig spelaren ofta igenom många rum och i ett rum är den akustiska vågekvationen approximativt linjär. Denna linjäritet ger möjligheten att superponera ljudkällor. När vi angriper vårt problem måste det bestämmas vilka randvillkor som skall väljas för väggarna i våra simulationer. Alla material är till en viss grad absorberande i den meningen att ljudvågen tappar energi när den träffar väggen. Skriver man vågekvationen i en dimension på operatorform
\begin{equation}
\left(\frac{\partial^{2}}{\partial t^{2}} - v^{2}\frac{\partial^{2}}{\partial x^{2}}\right)u = 0
\end{equation}
kan följande faktorisering göras
\begin{equation}
\left(\frac{\partial}{\partial t} - v\frac{\partial}{\partial x}\right)\left(\frac{\partial}{\partial t} + v\frac{\partial}{\partial x}\right)u = 0.
\end{equation}
Det visar sig att faktorerna är en endast högergående respektive en endast vänstergående våg. Detta utnyttjas för att skriva ett absorberande villkor som säger att om en vänstergående våg träffar randen kommer en skalad högergående våg att reflekteras
\begin{equation*}
u_{t} - vu_{x} = \alpha\left(u_{t} + vu_{x}\right) 
\end{equation*}
\begin{equation} \label{eq:abc}
\Rightarrow u_{t} - v\left(\frac{1 + \alpha}{1 - \alpha}\right)u_{x} = 0, \qquad \alpha \in [0,1).
\end{equation}
Reflektionsparametern $\alpha$ kontrollerar hur mycket vågen reflekteras.
Då $\alpha \rightarrow 0$ så fås $u_{t} - vu_{x} = 0$ d.v.s. perfekt absorberande randvillkor. Då $\alpha \rightarrow 1$ kommer $u_{x}$-termen dominera och då fås $u_{x} = 0$, det klassiska Neumannvillkoret som ger full reflektion.\\*
\newline